% ====================================================================
%+
% NAME:
%    short_exposures.tex
%
% CHAPTER:
%    specialsurveys.tex
%
% ELEVATOR PITCH:
%
% AUTHORS:
%    Chris Stubbs (@astrostubbs))
%-
% ====================================================================

\section{Short Exposure Surveying}
\def\secname{shortexp}\label{sec:\secname}

\credit{astrostubbs}

The current LSST requirements stipulate a minimum exposure time of 5
seconds, with an expected default exposure time of 15 seconds. This
document advocates for decreasing the minimum exposure time requirement
from 5 to 0.1 seconds. This would increase the dynamic range for bright
sources (compared to the default 15 sec time) by about 5 magnitudes, to
a total of 13 astronomical magnitudes (where dynamic range is the
difference between the brightest unsaturated source and the faintest
point source detectable at 5 sigma). This is a large factor, and would
enable a wide range of science goals, outlined below. One interesting
aspect of this is that it would allow us to operate the LSST system
during twilight times that would otherwise saturate the array due to
background sky brightness. This would allow a number of the goals
described below to be carried out without impacting the primary survey
by conducting observations during twilight sky conditions that would
saturate the array at longer exposure times.


% ----------------------------------------------------------------------

\subsection{Introduction}
\label{sec:\secname:intro}

Since the twilight sky brightness is an important factor discussed
below, we provide here a very brief outline of the temporal evolution of
the background sky brightness.

\citet{1993AJ....105.1206T}
provide a simple framework that serves our purposes well. They provide
observational data as well as a simple model for the evolution of
twilight sky brightness. Figure~1 from that paper is included below, as \autoref{fig:Tyson}.
They show that a good model for the sky brightness evolution is given by
an exponential with
$\log_{10}(S)=(k/\tau)t+C$,
where S is the sky brightness in electrons per pixel per second, C is
the dark sky background, k = (10.6 minutes)$^{-1}$  is a universal
(band-independent) timescale during which the sky's surface brightness
changes by a factor of ten (at latitude $-$30 degrees), and $\tau$ is a
season-dependent factor that ranges from 1.0 at the equinox to 1.07 in
austral winter and 1.20 in austral summer. So the rule of thumb is that
we should expect it to take 4.25 minutes for the sky background to
change by one magnitude per square arc sec. (In what follows we'll
ignore the increased twilight time in summer and winter.)

For current generation typical astronomical camera systems that take
over a minute to read out, this 4.2 minute time scale means that only a
handful of images can be obtained during twilight time. But for the LSST
camera with a 2 second readout time, we can obtain hundreds of short
exposures during twilight. Even if we are limited to a 15 second cadence
due to thermal stability or data transfer limitations there is a large
amount of time opened up that we can use.

What do we stand to gain in operational time with shorter exposures? If
the standard survey terminates taking 15 second exposures due to some
sky brightness criterion, by shifting to 0.1 sec images at that point we
will have changed the sky flux per pixel by 2.5 $\log_{10}(150)$ = 5.4
magnitudes. This brings us back into a high dynamic range regime, as
described below.

\begin{figure}[htbp]
\begin{center}
\includegraphics[trim = 0 7cm 0 1mm, clip, width=\textwidth]{figs/Stubbs_Fig1.pdf}
\caption{(reproduced from Tyson et al, 1993). This plot shows the
  twilight sky surface brightness as a function of local time for four
  broadband filters (C, B, V and R) and different pointing directions.
  The surface brightness changes by one magnitude in a 4.2 minute interval,
essentially independently of the passband and pointing.}
\label{fig:Tyson}
\end{center}
\end{figure}

\autoref{fig:twilight} illustrates the principles that underpin this proposal. LSST is
a unique combination of hardware and software, that will deliver
reliable catalogs of both the static and the dynamic sky. By pushing
towards shorter integration times we can greatly expand the scientific
reach of the system.

The dynamic range in magnitudes that we can achieve for a given
integration time depends on the sky background, the read noise, and the
full well depth per pixel. We will adopt a typical value of 100Ke for
the full well depth, but the arguments presented below are essentially
independent of this value. The dynamic range in magnitudes is limited on
the bright end by the point source whose PSF peak exceeds full well, and
on the faint end by the 5$\sigma$ point source sensitivity, which
depends on sky brightness per pixel. So we are squeezed between the two
parameters of full well depth and sky background.

\begin{figure}[htbp]
\begin{center}
\includegraphics[width=6in]{figs/Stubbs_Fig2.pdf}
\caption{Twilight dynamic range. As we enter morning twilight time, the increasing sky brightness requires brighter sources for 5 sigma detection, and also limits unsaturated objects to increasingly fainter sources. Eventually the gap between these goes to zero. But operating at shorter exposure times allows us to push useful survey operations into brighter twilight time, and also to increase the dynamic range of the LSST survey products. The black lines correspond to 15 second integrations (nominally in the r band), the red lines to 5 second exposures, and the blue curves to 0.1 second exposures. The upper lines in each case represent the 5 sigma point source detection threshold while the lower line corresponds to the source brightness that produces saturation in the peak pixel of the PSF. Adding shorter exposure times increases our dynamic range in flux, and adds valuable observing time.}
\label{fig:twilight}
\end{center}
\end{figure}

The 5-sigma limiting flux scales as the square root of the sky
brightness, while the saturation flux decreases linearly as sky
brightness increases. So the two curves in \autoref{fig:twilight} have
slopes that differ by a factor of two. Operating during bright-sky time
with short exposures adds about 20 minutes of observing per twilight, or
40 minutes per night. This is a non-trivial resource!

\autoref{fig:twilight} shows one reason why it is not advantageous to go
below 0.1 second exposures- we would lose the overlap between a twilight
survey and the standard LSST object catalog.


% ----------------------------------------------------------------------

\subsection{Science Drivers for Shorter Exposures}
\label{sec:\secname:drivers}

Having set the stage for the opportunity to operate at shorter exposure
times either during dark sky time, or during twilight, or both, we now
describe some of the scientific motivations for doing so.


\subsubsection{Discovery space at short time scales.}

LSST is a time domain discovery machine. It is hard to anticipate the
importance of being able to detect astronomical variability on short
time scales. By extending the time domain sensitivity to phenomena with
a characteristic time of less than 5 seconds, we will have added 1.5
orders of magnitude in time domain sensitivity.

Taking short exposures does not necessarily imply a requirement on fast
image cadence. Periodic variability can be readily detected and
characterized with a succession of short images that do not satisfy the
Nyquist criterion, as long as we know the time associated with each data
point to adequate accuracy. But it does seem appropriate to investigate
the maximum possible rapid-fire imaging rate for LSST, presumably
limited by either data transfer bottlenecks or by thermal issues within
the camera.

\subsubsection{Distances to Nearby SN Ia- an essential ingredient in using supernovae to probe dark energy.}

The determination of the equation of state parameter of the Dark Energy
using type Ia supernovae entails measuring the redshift dependence of
the luminosity distances to objects over a range of redshifts. The low
end of this redshift range is limited by peculiar velocities to
considering supernovae at redshifts z$>$0.01. At this distance (distance
modulus of $\mu$ =33) the peak brightness of a type Ia supernova is r=15
and exceeds the expected LSST point source saturation limit.

Moreover, the rate on the sky of these bright nearby supernovae is so
low that in the standard cadence we don't expect to obtain well-sampled
multiband light curves for them. But we will discover many of them on
the rise. Using twilight time with short exposures to obtain appropriate
temporal and passband coverage will allow us to extend the LSST SN
Hubble diagram across the entire redshift range of 0.01 to 1.

It is vitally important that we obtain these nearby-SN light curves on
the same photometric system, reduced with the same data reduction
pipeline, as the distant sample. This means we really must use the LSST
instrument and software in order to avoid systematic errors arising from
differences in photometric systems or algorithmic issues.

We stress that this twilight SN followup campaign can be accomplished
without impacting the main survey, during the roughly 20 minutes per
night of twilight that would otherwise unusable at the default exposure
time. We would use the brighter twilight time to obtain pointed
observations on nearby supernovae, motivated by the importance of
photometric uniformity described above.


\subsubsection{A Bright Star Survey for Galactic Science.}

We could also use the added twilight time to conduct a bright star
survey, and the precise astrometry and photometry from LSST can then be
used in conjunction with archived data ranging from 11th to 27th AB
magnitudes. This short-exposure domain would extend the LSST dynamic
range in fluxes by two orders of magnitude, towards the bright end.
Moreover, obtaining precise positions, fluxes and variability at these
brighter magnitudes would greatly increase the overlap with the
historical archive of astronomical information, including from digitized
plate data. We would be able to obtain astrometric and color information
to high precision, as well time series for variability studies.

An example of an application to Milky Way structure studies comes from
RR Lyrae variable stars. With a saturation magnitude of around 16th in
the standard LSST survey, RR Lyrae closer than 20 kpc will be saturated
in the standard LSST images. So we will lose nearly all Galactic RR
Lyrae. Extending the survey's bright limit to 11th magnitude will allow
us to collect light curves for RR Lyrae beyond $\sim$ 100 parsecs,
collecting essentially all Southern hemisphere Galactic RR Lyrae.

Another application for stellar population studies is measuring the
fraction of binary stars as a function of stellar type, metallicity, age
and environment. By conducting a variability survey in the 11-18
magnitude range we can capitalize on temperature and metallicity data
already in hand for many of these objects.

Another application of a bright star survey would be to search for
planetary transits in the magnitude range appropriate for radial
velocity followup observations using 30 meter class telescopes. For high
dispersion spectrographs at the 4m aperture class, most targets are
currently around 8th magnitude, so we should expect 30m telescopes to
attain similar radial velocity precisions for sources of magnitude  8 +
5log(30/4) = 12. By going to shorter exposures we obtain almost an
hour's additional observing time per night when these sources don't
saturate, whereas they are far beyond saturation in the default 15
second LSST survey images.

A typical (r$-$K) color between SDSS and 2MASS is r$-$K=3. The 2MASS
catalog is complete down to K$\sim$14 which corresponds to r$\sim$17. So
most 2MASS stars will be saturated in the standard LSST 15 second
observations. A bright star survey will allow a multiband match to the
2MASS data, as well as an astrometric comparison between the two
catalogs.

Finally, the apparent magnitude of solar system objects depends on their
distance from us and from the sun, as well as illumination and
observation geometry. Extending the bright limit will allow us to track
asteroid positions as they approach opposition.


% ----------------------------------------------------------------------

\subsection{Counterarguments}
\label{sec:\secname:counter}


\subsubsection{What About Scintillation Effects?}

Short exposure times suffer from scintillation effects. An estimate for
uncertainty due to scintillation is provided by
\url{http://astro.corlan.net/gcx/scint.txt}. For a 0.1 second
integration we expect a fractional flux uncertainty of  0.15 at 2
airmasses and 0.043 at 1 airmass, for a 10 cm aperture. Scaling this up
to the 8.5m aperture of LSST by a factor D$^{2/3}$ predicts fractional
flux variations of below one percent, even at two airmasses, for a 0.1
second exposure. So scintillation should not impact our ability to make
precision measurements of flux and position.

\subsubsection{What about just doing this with smaller telescopes?}

A possible counter-argument to the proposal of allowing for shorter
exposure times is that much of this can be done with smaller telescopes.
But it's important to bear in mind that LSST is a system, and the data
reduction and dissemination tools are as important as the hardware. We
intend to deliver accessible, high-quality, well-calibrated photometry
on a common photometric system and correspondingly good positions. If we
do so from a co-added point source depth of 27th to the short-exposure
bright limit of 11th magnitude we will span over six decades in flux on
a well-calibrated flux scale. We would also have the ability to study
astrophysical variability on time scales from 0.1 second to 10 years,
which is nine decades in the time domain. This combination of temporal
and flux dynamic range would be a truly remarkable  achievement, and
would yield science benefits far beyond the illustrative examples
provided above. Much of this discovery space is enabled by going to
shorter exposures.

\subsection{Proposed Implementation and Impacts}

The implementation of this would simply entail taking short-exposure
images during twilight time that would otherwise go unused. The data
rate would go up, and the number of shutter cycles per night would also
increase.

%
%\section{References}
%
%Tyson and Gal, An Exposure Guide for Taking Twilight Flats with Large Format CCDs, AJ {\bf 105}, 1026 (1003).
