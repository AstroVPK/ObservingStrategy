% ====================================================================
%+
% NAME:
%    section-name.tex
%
% ELEVATOR PITCH:
%    Explain in a few sentences what the relevant discovery or
%    measurement is going to be discussed, and what will be important
%    about it. This is for the browsing reader to get a quick feel
%    for what this section is about.
%
% COMMENTS:
%
%
% BUGS:
%
%
% AUTHORS:
%    David Nidever (@dnidever)
%    Knut Olsen (@knutago)
%-
% ====================================================================

\section{Solar System mini-surveys}
\def\secname{solar_system_specials}\label{sec:\secname}

\credit{davidtrilling},
\credit{rhiannonlynne}.

% This individual section will need to describe the particular
% discoveries and measurements that are being targeted in this section's
% science case. It will be helpful to think of a ``science case" as a
% ``science project" that the authors {\it actually plan to do}. Then,
% the sections can follow the tried and tested format of an observing
% proposal: a brief description of the investigation, with references,
% followed by a technical feasibility piece. This latter part will need
% to be quantified using the MAF framework, via a set of metrics that
% need to be computed for any given observing strategy to quantify its
% impact on the described science case. Ideally, these metrics would be
% combined in a well-motivated figure of merit. The section can conclude
% with a discussion of any risks that have been identified, and how
% these could be mitigated.

%A short preamble goes here. What's the context for this science
%project? Where does it fit in the big picture?

There are several populations of Near Earth Objects (Solar System bodies
whose orbits bring them close to the Earth's orbit) that, because of
their orbital properties, would not be easily detected in the
wide-fast-deep survey. These populations are very interesting for both
scientific and sociological purposes, though, due to their close
proximity to the Earth, and in fact their potential for impacting the
Earth. LSST will have the capability to carry out surveys for these
populations by using a small amount of time in ``mini-surveys.'' Two of
these mini-surveys have pointings that fall within the nominal
wide-fast-deep plan, and simply require a modification of the cadence.
The third program is a twilight program, with a special cadence (though
all twilight programs are likely to  have special cadences). These three
programs are listed here and described below. The three mini-surveys are
the following:

\begin{itemize}
\item A mini-survey to look for mini-moons, which are temporarily captured
satellites of the Earth;
\item A mini-survey to find meter-sized impactors up to two weeks prior to impact.
This would allow telescopic characterization of these impactors, which could
be compared to laboratory measurements of the meteorites derived from
the impactor. Advanced warning of an impactor also allows detailed
study of impact physics by being on location when the impact
occurs;
\item A mini-survey to observe the ``sweetspot'' in twilight fields
to look for NEOs in very Earth-like orbits that would otherwise not
be found in opposition fields.
\end{itemize}

% Need the figure and caption
These surveys will support two important scientific investigations:
\begin{enumerate}
\item What are the properties of the population of objects that is
nearest to the Earth?
\item What is the impact risk from NEOs in populations that
have not yet been well characterized (mini-moons, sweetspot objects)?
\item How do the telescopic properties of an impactor relate to the
laboratory-measured properties of the ensuing meteorites?
\end{enumerate}

Many different types of objects and measurements with their own cadence
``requirements'' will fall into these two broad categories (with some
overlap).  These will be outlined in the next section.

% --------------------------------------------------------------------

\subsection{Target measurements and discoveries}
\label{sec:\secname:targets}

\subsubsection{Special cadences}

Each of the three Solar System mini-surveys requires a special
cadence. These cadences are described here.

\begin{itemize}

\item{{\bf Mini-moons}}
Mini-moons are objects that are temporary satellites of the Earth.
Therefore, they have orbital motions similar to the Earth's moon,
and much faster than other Solar System populations. Therefore,
a special cadence is required to detect these objects enough
times to link objects, create tracklets, and determine orbits.
A suggested cadence for a mini-moon survey is a series
of 3~second exposures, with each pointing visited at least
twice per night. Such a survey would cover essentially
all of the opposition sky each night. The opposition sky should
be re-observed several nights in a row in order to
link objects from night to night and determine their orbits.
xxx need to work on these details; a bit sketchy right now xxx

\item{{\bf Impactors}}
The Earth is struck by meter-sized impactors about
once a month xxx confirm xxx.
On two occasions, impacting asteroids have
been discovered some hours before impact, but
there are no existing surveys that are dedicated to finding
impactors.
% xxx ATLAS xxx.
Impactors generally have small apparent motions
on the sky (because their orbits are not too different
than the Earth's). The single exposure depth of LSST
images suggests that a meter-sized NEO could be
discovered perhaps a week before impact (given
the typical Earth-relative velocity of such a body).
A suggested cadence for an impactor survey would be
to survey the opposition patch four times per night.
This is more visits than in the nominal cadence, and
would allow high fidelity linking of observations to
find orbits. The nominal wide-fast-deep cadence
(twice per night, three times during a lunation) has
a latency of orbit determination of up to two weeks,
which is not acceptable for the impactor survey, as an
impact would occur in a timescale of just a few days
from discovery.
The cadence of four observations/night should be repeated
roughly every three days, so that an object on an
inbound trajectory could be observed at least once,
and possibly twice, before impact.
Note that this cadence is compatible with
the wide-fast-deep survey, in that the fields and
exposure times are nominal; the only difference is that
each field is visited four times in a night, and that
the fields are revisited every few nights. The overall
impact of this mini-survey on the wide-fast-deep
survey is likely to be small, and possibly negligible.

\item{{\bf Twilight/sweetspot survey}}

NEOs on very Earth-like orbits are relatively
unlikely to come to opposition, and therefore
are relatively unlikely to appear in data
obtained in the wide-fast-deep survey.
These objects are particularly interesting
since, having very Earth-like orbits, they
are the most likely objects to be Earth
impactors.
These objects are most likely to be detected
in a twilight survey that looks at the ``sweetspot'' ---
a location at around 60~degrees Solar
elongation that is only visible at twilight.
Because these sweetspot fields are only visible
for a small amount of time in a night, a special
cadence is required to find and link these objects
to determine their orbits.
These observations would be best carried out
in the z filter (because the observations are
made in twilight, when the sky is still relatively
bright). Fields should be revisited at 15~minute
intervals, and each field should be revisited
every other night during this experiment, so that
observations can be linked.
(A long interval
between observations prohibits linking.)
The total experiment
should last roughly one week, so that each
object would have a tracklet on four nights
(nights 1,3,5,7).
During twilight, some 25~pointings could be visited
before the fields have set.
Because these observations are made during twilight,
there may be no significant impact on the
nominal wide-fast-deep survey.
\end{itemize}

\subsubsection{Measurements}

For each of these three programs, the most important measurement
to be made is the position of any object as a function of time.
In other words, the usual measurements of moving
objects from LSST images is also the requirement for
the source detections for these mini-surveys. As usual
for Solar System surveys, there is a trade-off of
sensitivity (Solar System objects are most easily
detected in r band) against characterization (observing
a given object in multiple filters yields an estimate
of composition). For these three cases, discovery and
good orbit determination is probably more important than
immediate characterization from LSST measurements,
so the nominal expectation is that all these mini-surveys
would be carried out in r band.


% --------------------------------------------------------------------

\subsection{Metrics}
\label{sec:\secname:metrics}

The metrics to be used to determine the efficacy of LSST
at scientific success of these mini-surveys are identical
to those employed in \autoref{chp:solarsystem}.
The most important of these metrics
include the completeness as a function of size; the
number of detections over a given length of time (for instance,
the one week approach timescale of impactors); and
the quality of the derived orbit. These metrics are defined
in more detail in \autoref{chp:solarsystem}. The important question is:
how much value do the mini-surveys add?


% --------------------------------------------------------------------

\subsection{OpSim Analysis}
\label{sec:\secname:analysis}

The current default observing strategy does not include
any of these mini-surveys. Therefore, the scientific yield,
at this default, is zero. Both the mini-moons and impactor
surveys are relatively small experiments, on the scale of
the LSST project, at something like 10--20~hours total
per instance of the experiment. (The impactor experiment,
for example, might be carried out one or several times a year,
both to build up statistics and to identify further potential
impactors.) Furthermore, the impactors survey cadence
is different from the nominal wide-fast-deep survey,
but could be a simple modification of the nominal wide-fast-deep survey
cadence.

The twilight/sweetspot survey is also not included in
the nominal OpSim strategy, and the overall discussion
of twilight observations is deferred to a later discussion.

%
% % --------------------------------------------------------------------
%
% \subsection{Discussion}
% \label{sec:\secname:discussion}
%
% Discussion: what risks have been identified? What suggestions could be
% made to improve this science project's figure of merit, and mitigate
% the identified risks?
%

% ====================================================================

\navigationbar
