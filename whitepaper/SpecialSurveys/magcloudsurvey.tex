% ====================================================================
%+
% NAME:
%    section-name.tex
%
% ELEVATOR PITCH:
%    Explain in a few sentences what the relevant discovery or
%    measurement is going to be discussed, and what will be important
%    about it. This is for the browsing reader to get a quick feel
%    for what this section is about.
%
% COMMENTS:
%
%
% BUGS:
%
%
% AUTHORS:
%    David Nidever (@dnidever)
%    Knut Olsen (@knutago)
%-
% ====================================================================

\section{The Magellanic Clouds Special Survey}
\def\secname{mc}\label{sec:\secname}

\credit{dnidever},
\credit{knutago}.


An LSST survey that did not include coverage of the Magellanic Clouds
and their periphery would be tragically incomplete.  LSST has a unique
role to play in surveys of the Clouds.  First, its large $A\Omega$
will allow us to probe the thousands of square degrees that comprise
the extended periphery of the Magellanic Clouds with unprecedented
completeness and depth, allowing us to detect and map their extended
disks, stellar halos, and debris from interactions that we already
have strong evidence must exist (REFS).  Second, the ability of LSST
to map the entire main bodies in only a few pointings will allow us to
identify and classify their extensive variable source populations with
unprecedented time and areal coverage, discovering, for example,
extragalactic planets, rare variables and transients, and light echoes
from explosive events that occurred thousands of years ago (REFS).
Finally, the large number of observing opportunities that the LSST
10-year survey will provide will enable us to produce a static imaging
mosaic of the main bodies of the Clouds with extraordinary image
quality, an invaluable legacy product of LSST.

% --------------------------------------------------------------------

\subsection{A Proposed Magellanic Clouds Mini-survey}
\label{sec:\secname:proposal}

We propose two distinct mini-surveys to meet the goals of LSST
Magellanic Clouds science:
\begin{itemize}
\item A mini-survey covering the 2700$\deg^2$ with $\delta < -60$ to
the standard LSST single-exposure depth and to stacked depths of XXX,
with cadence sufficient to detect and measure light curves of RR Lyrae
stars.
\item A mini-survey covering $\sim$250$\deg^2$ of the main bodies of
the Clouds with cadence sufficient to detect exoplanet transits and
other variable objects; a subset of these images should be taken with
seeing of $0.5\arcsec$, with stacked depth reaching the confusion
limits in the Clouds.
\end{itemize}

Figure X shows a rough map of the proposed mini-surveys.
% Need the figure and caption


% --------------------------------------------------------------------

\subsection{Mini-survey Impact on the Magellanic Cloud Science Projects}
\label{sec:\secname:revisit}

\new{Here we revisit the metric analysis of the Magellanic  Clouds'
science cases (\autoref{chp:MCs}), and make some predictions about how
they are likely to improve given  the above proposal.}


% --------------------------------------------------------------------

\subsection{Discussion}
\label{sec:\secname:discussion}


% ====================================================================

\navigationbar
