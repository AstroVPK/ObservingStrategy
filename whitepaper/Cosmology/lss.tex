% ====================================================================
%+
% SECTION NAME:
%    lss.tex
%
% CHAPTER:
%    cosmology.tex
%
% ELEVATOR PITCH:
%    Large Scale Structure, Weak Lensing, and Clusters all require
%    survey uniformity in the static 10-year survey.  A key contributor
%    to this is the pattern of dithers adopted.
%
% AUTHORS:
%   Eric Gawiser (@egawiser)
%-
% ====================================================================

\section{Large Scale Structure:  Testing Dithering Patterns and Timescales to Improve Survey Uniformity}
\def\secname{lss}\label{sec:\secname}

\credit{HumnaAwan},
\credit{egawiser}
\credit{PeterKurczynski},
\credit{rhiannonlynne}

Three of the key cosmology probes available with LSST represent ``static
science'' insensitive to time-domain concerns.  These are Weak Lensing,
Large-Scale Structure, and Galaxy Clusters.  Nonetheless, due to the
need to track and correct for the survey ``window function'' in all of
these probes, cosmology with LSST will benefit greatly from achieving
survey depth as uniform as possible over the WFD area.  OpSim tiles the
sky in hexagons inscribed within the nearly-circular LSST field-of-view.
It has been shown in \citet{CarrollEtal2014} that the default LSST
survey strategy implemented in OpSim runs leads to a strongly
non-uniform ``honeycomb'' pattern due to overlapping regions on the
edges of these hexagons receiving double the observing time.  A pattern
of large dithers proves sufficient to greatly reduce these overlaps,
leading to an increase in median survey depth in each filter of 0.08
magnitudes.

In this section, we report results from an investigation by Awan et al.
(in preparation) of several geometrical patterns for dithers performed
on timescales varying from once per observing season to once per night
to every visit.

% \todo{EG}{Flesh out WL, LSS, and Clusters dependence on survey
% uniformity to make this section more clearly science-driven.}

% --------------------------------------------------------------------

\subsection{Dithering Patterns and Timescales}
\label{sec:\secname:strategies}


% --------------------------------------------------------------------

\subsection{Metrics}
\label{sec:\secname:metrics}

% Quantifying the response via MAF metrics: definition of the metrics,
% and any derived overall figure of merit.

Our primary metric is the total uncertainty in the derived window function
over relevant angular scales, modeled via variations in the angular
power spectrum of fake galaxy fluctuations between $gri$ bands.
Intermediate metrics include the number of galaxies in each pixel,
fluctuations in this number, total power in the angular power spectrum
of a skymap of those fluctuations, and residual power that angular power
spectrum after subtracting a smooth fit to it.


% --------------------------------------------------------------------

\subsection{OpSim Analysis}
\label{sec:\secname:analysis}

In this section we present our ongoing \OpSim / MAF analysis, as we try
to answer the question ``what dithering strategies produce acceptable
variations in survey uniformity, and which appears optimal?''

What we show here are selected highlights from the paper by Awan et al (2015).

%We used the
%\simsMAFcontrib{SeasonStacker}{mafContrib/seasonStacker.py} to work
%with seasons.

%We used \texttt{ops2\_1075} for most of our tests, but we need to now
%re-run on \opsimdbref{db:baseCadence}, and others from \autoref{chp:cadence2015}.


%\citeauthor{LiaoEtal2015}). These sky maps show that, over the main

%\autoref{tab:lenstimedelays:results} shows the global (i.e. al-sky)


% %--------------------------------------------------------------------
%
% \subsection{Results}
% \label{sec:\secname:results}


% \todo{EG}{Improve figures to originals rather than screen-captures.}


%%%%%%%%%%%%%%%%%%%%%%%%%%%%%%%%%
\begin{figure}[tbh!]
\vskip -0.1in
\includegraphics[angle=0,width=0.99\hsize:,clip]{figs/awan_fig1.png}
%\vskip -1.3in
\caption{}
\label{fig:seasonal_dithers}
\end{figure}
%%%%%%%%%%%%%%%%%%%%%%%%%%%%%%%%%

%%%%%%%%%%%%%%%%%%%%%%%%%%%%%%%%%
\begin{figure}[tbh!]
\vskip -0.1in
\includegraphics[angle=0,width=0.99\hsize:,clip]{figs/awan_fig2.png}
%\vskip -1.3in
\caption{}
\label{fig:nightly_dithers}
\end{figure}
%%%%%%%%%%%%%%%%%%%%%%%%%%%%%%%%%

%%%%%%%%%%%%%%%%%%%%%%%%%%%%%%%%%
\begin{figure}[tbh!]
\vskip -0.1in
\includegraphics[angle=0,width=0.99\hsize:,clip]{figs/awan_fig4.png}
%\vskip -1.3in
\caption{}
\label{fig:dithering_histograms}
\end{figure}
%%%%%%%%%%%%%%%%%%%%%%%%%%%%%%%%%

%%%%%%%%%%%%%%%%%%%%%%%%%%%%%%%%%
\begin{figure}[tbh!]
\vskip -0.1in
\includegraphics[angle=0,width=0.99\hsize:,clip]{figs/awan_fig5.png}
%\vskip -1.3in
\caption{}
\label{fig:dithering_skymaps}
\end{figure}
%%%%%%%%%%%%%%%%%%%%%%%%%%%%%%%%%

%%%%%%%%%%%%%%%%%%%%%%%%%%%%%%%%%
\begin{figure}[tbh!]
\vskip -0.1in
\includegraphics[angle=0,width=0.99\hsize:,clip]{figs/awan_fig6.png}
%\vskip -1.3in
\caption{}
\label{fig:dithering_power_spectra}
\end{figure}
%%%%%%%%%%%%%%%%%%%%%%%%%%%%%%%%%




%%%%%%%%%%%%%%%%%%%%%%%%%%%%%%%%%%%%
%\begin{figure*}[!ht]
%  \capstart
%  \begin{minipage}[b]{\linewidth}
%    \begin{minipage}[b]{0.32\linewidth}
%      \centering\includegraphics[width=\linewidth]{figs/Accuracy_season_nca.pdf}
%    \end{minipage} \hfill
%    \begin{minipage}[b]{0.32\linewidth}
%      \centering\includegraphics[width=\linewidth]{figs/Precision_cadence_nca.pdf}
%    \end{minipage} \hfill
%    \begin{minipage}[b]{0.32\linewidth}
%      \centering\includegraphics[width=\linewidth]{figs/Fraction_season_nca.pdf}
%    \end{minipage}
%  \end{minipage}
%\caption{Examples of changes in accuracy $A$ (left), precision $P$ (center) and success fraction $f$ (right) with schedule properties, as seen in the different TDC submissions. The gray
%approximate power law model was derived by visual inspection of the
%pyCS-SPL results; the signs of the indices were pre-determined according to our expectations. Reproduced from \citet{LiaoEtal2015}.}
%\label{fig:tdcresults}
%\end{figure*}
%%%%%%%%%%%%%%%%%%%%%%%%%%%%%%%%%%%


% \todo{EG}{Input fuller results and text from Awan et al. draft.}

% %---------------------------------------------------------------------
%
% \subsection{Figure of Merit}
% \label{sec:\secname:fom}
%
%

% % --------------------------------------------------------------------
%
% \subsection{Discussion}
% \label{sec:\secname:discussion}
%
% Discussion: what risks have been identified? What suggestions could be
% made to improve this science project's figure of merit, and mitigate
% the identified risks?

\navigationbar

% ====================================================================
