% --------------------------------------------------------------------

\chapter[Stellar and Binary Variables]{Stellar and Binary Variables}
\def\chpname{variables}\label{chp:\chpname}

Chapter editors:
\credit{AshishMahabal},
\credit{lmwalkowicz}.

% \noindent {\it
% Mike Lund, Ashish Mahabal, Stephen Ridgway,
% Lucianne Walkowicz, Rahul Biswas, Michelle Lochner,
% Jeonghee Rho, Eric Bellm...
% }

% --------------------------------------------------------------------


\section{Introduction}

The observation of variable targets investigates the time domain, and
thus realizes the essence of a synoptic survey.  Variable and
transient studies are all about sampling, and different types and time
scales of phenomena benefit from different sampling strategies -
sometimes significantly different.  Competing objectives described in
this chapter are at the heart of LSST observing strategy and cadence
design.

Target types are here grouped in subsections by variability
characteristics, but as will be seen, this does not mean that all
targets in a group require a common cadence, since the times scales
may vary dramatically.  Acquiring suitable data for a wide range of
time scales presents a fundamental problem for LSST, since the
available $~$800 visits to a field over the survey cannot be deployed
so as to usefully sample all time scales at all times.  This fact
leads to the concept of a non-uniform survey, in which parts of the
sky are visited more frequently part of the time.  The merits of such
options must be traded against the benefits of a more uniform survey
strategy.

When evaluating a particular observation or series of observations in
light of how they perform for a specific science case, it may be
helpful to think of metrics as lying along a continuum between
discovery and characterization. Discovery requires a minimum amount of
information to recognize an event or object as a candidate of
interest, which necessarily involves some level of bare-bones
characterization (upon which said recognition is based); rich
characterization, on the other hand, implies that an event may not
only be recognized as a candidate of interest, but basic properties of
the event or object may be determined from the observation (e.g.
including but not limited to classification of the event). The
interpretation of a given metric along this continuum has implications
for the subsequent action and analysis required, particularly as
regards possible follow-up observations with other facilities.


% --------------------------------------------------------------------


% ====================================================================
%+
% NAME:
%    section-name.tex
%
% ELEVATOR PITCH:
%    Explain in a few sentences what the relevant discovery or
%    measurement is going to be discussed, and what will be important
%    about it. This is for the browsing reader to get a quick feel
%    for what this section is about.
%
% COMMENTS:
%
%
% BUGS:
%
%
% AUTHORS:
%    Phil Marshall (@drphilmarshall)  - put your name and GitHub username here!
%-
% ====================================================================

\section{Discovery of Periodic Pulsating Variables}
\def\secname{periodicvariables}\label{sec:\secname}

\noindent{\it Lucianne M. Walkowicz, \&c} % (Writing team)

% This individual section will need to describe the particular
% discoveries and measurements that are being targeted in this section's
% science case. It will be helpful to think of a ``science case" as a
% ``science project" that the authors {\it actually plan to do}. Then,
% the sections can follow the tried and tested format of an observing
% proposal: a brief description of the investigation, with references,
% followed by a technical feasibility piece. This latter part will need
% to be quantified using the MAF framework, via a set of metrics that
% need to be computed for any given observing strategy to quantify its
% impact on the described science case. Ideally, these metrics would be
% combined in a well-motivated figure of merit. The section can conclude
% with a discussion of any risks that have been identified, and how
% these could be mitigated.

Regular variables, such as Cepheids and RR Lyraes, are valuable tracers of Galactic structure and cosmic distance. In this case of these and other strictly (or nearly-strictly) periodic variables, data from different cycles of observation can be phase-folded to create a more fully sampled lightcurve as LSST visits will occur effectively at random phases. In a 10-year survey, most periodic stars of almost any period will benefit from excellent phase coverage in all filters (only a very small period range close to the sidereal day will be poorly observed). Therefore, most implementations of the LSST observing strategy will provide good sampling of periodic variables.

However, different implementations of the survey may result in different resulting sample sizes of these periodic variables, and may also affect the environments in which these stars are discovered. In this section, we create a framework for understanding how current implementations of the observing strategy influence (or even bias) the resultant sample size and environments where these important tracers may be identified. 

\subsection{Tracing Galactic Structure with RR Lyrae}

Oluseyi et al. 2012 [INSERT REF] carried out an extensive simulation of period and lightcurve shape recovery of RR Lyrae variables. Correctly identifying the period aids in building the sample of interest, whereas a more finely sampled lightcurve can make it possible to measure the lightcurve shape, which makes it possible to measure the metallicity of the star.

- lightcurve template matching (shape recovery metric) 

% --------------------------------------------------------------------

\subsection{The Cepheid Cosmic Distance Ladder}

% --------------------------------------------------------------------

\subsection{OpSim Analysis}
\label{sec:keyword:analysis}

Current simulations of the main survey show a broad uniformity of visits, with thorough randomization of visit phase per period, giving very good phase coverage with minimum phase gaps.


% --------------------------------------------------------------------

\subsection{Discussion}
\label{sec:keyword:discussion}

For periodic variable science, two cadence characteristics should be avoided:
\begin{itemize}
\item an exactly uniform spacing of visits (which is anyway virtually impossible); \
\item a very non-uniform distribution, such as most visits concentrated in a few survey years.
 \end{itemize}

A metric for maximum phase gap will guard against the possibility that a very unusual cadence might compromise the random sampling of periodic variables.

In each case, it would help to jump-start science programs if some fraction of targets had more complete measurements early in the survey.


% ====================================================================
\navigationbar


% --------------------------------------------------------------------

\input{nonperiodicvariablestars.tex}

% --------------------------------------------------------------------
