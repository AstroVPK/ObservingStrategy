% ====================================================================
%+
% NAME:
%    planets.tex
%
% CHAPTER:
%    variables.tex
%
% ELEVATOR PITCH:
%-
% ====================================================================

% \section{Probing Planet Populations with LSST}
\subsection{Probing Planet Populations with LSST}
\def\secname{planets}\label{sec:\secname}

\credit{lundmb},
\credit{shporer},
\credit{stassun}

This section describes the unique discovery space for
extrasolar planets with LSST, namely,
planets in relatively unexplored environments.

% \subsubsection{Planets In Relatively Unexplored Environments}

A large number of exoplanets have been discovered over the past few
decades, with over 1500 exoplanets now confirmed. These discoveries are
primarily the result of two detection methods: The radial velocity (RV)
method where the planet's minimum mass is measured, and the transit
method where the planet radius is measured and RV follow-up allows the
measurement of the planet's mass and hence mean density. Other methods
are currently being developed and use to discover an increasing number of
planets, including the microlensing method and direct imaging. In
addition, the Gaia mission is expected to discover a large number of
planets using astrometry \citep{2014exha.book.....P}.

The {\it Kepler} mission has an additional almost 4000 planet
candidates. While these planet candidates have not been confirmed, the
sample is significant enough that planet characteristics can be studied
statistically, including radius and period distributions and planet
occurrence rates. LSST will extend previous transiting planet searches
by observing stellar populations that have generally not been
well-studied by previous transiting planet searches, including star
clusters, the galactic bulge, red dwarfs, white dwarfs (see below), and
the magellanic clouds (see Section~7). Most known exoplanets have been
found relatively nearby, as exoplanet systems with measured distances
have a median distance of around 80~pc, and 80\% of these systems are
within 320~pc (exoplanets.org). LSST is able to recover transiting
exoplanets at much larger distances, including in the galactic bulge and
the Large Magellanic Cloud, allowing for measurements of planet
occurrence rates in these other stellar 
environments \citep{2015AJ....149...16L,2015AJ....150...34J}.
Red dwarfs have often been
underrepresented in searches that have focused on solar-mass stars, however red
dwarfs are plentiful, and better than 1 in 7 are expected to host earth-sized
planets in the habitable zone \citep{2015ApJ...807...45D}.

Another currently unexplored environment where LSST will be able to
probe the exoplanet population is planets orbiting white dwarfs (WDs).
Such systems teach us about the future evolution of planetary systems
with main-sequence primaries, including that of the Solar System. When a
WD is eclipsed by a planet (or any other faint low-mass object,
including a brown dwarf or a small star) the radius and temperature
ratios lead to a very deep eclipse, possibly a complete occultation,
where during eclipse the target can drop below the detection threshold.
The existence of planets orbiting WDs has been suggested
observationally
\citep[e.g.,][]{2009ApJ...694..805F,2009AJ....137.3191J,2010ApJ...722..725Z,2012ApJ...747..148D}.
and theoretically \citep[e.g.,][]{2010MNRAS.408..631N}.
A few brown dwarf companions were already discovered
\citep[e.g.,][]{2006Natur.442..543M,2012ApJ...759L..34C,2006Sci...314.1578L,2014MNRAS.445.2106L},
and \citet{2015Natur.526..546V}
recently discovered a disintegrating planetary body orbiting a WD
\citep[see also][]{2015arXiv151006434C,2016ApJ...818L...7G,2016MNRAS.458.3904R}.

While most of the sky that LSST will survey will be at much lower
cadences than transiting planet searches employ, a sufficient
understanding of the LSST efficiency for detecting planets combined with
the large number of targets may still provide significant results.
Additionally, the multiband nature of LSST provides an extra benefit, as
exoplanet transits are achromatic while many potential astrophysical
false positives, such as binary stars, are not.
Indeed, as demonstrated by \citet{2015AJ....149...16L}, the multi-band LSST light curves 
can likely be combined to create merged light curves with denser sampling and effectively 
higher cadence, enabling detection of transiting exoplanets. The deep-drilling fields in
particular should prove to be a rich trove of transiting exoplanet detections, with 
transit-period recoverability rates as high as $\sim$50\% or more among Hot Jupiters around 
solar-type stars out to distances of many kpc and even the Magellanic Clouds in some cases 
\citep{2015AJ....149...16L,2015AJ....150...34J}.
Yields may be expected perhaps as early as the third year of LSST operations 
(Jacklin et al., in prep).
The ability to detect transiting planets outside of the deep-drilling fields is less certain;
here the details of the cadence among the various passbands will likely be particularly
important to assess carefully.


\subsection{Metrics}
\label{sec:\secname:metrics}
The detection of transiting planets will be dependent on having observations
that will provide sufficient phase coverage for transiting planets, with periods
that can range from less than one day up to tens of days. In order to address
this range of periods, an initial metric that can be used to address the detection
of transiting planets is the Periodogram Purity Function, discussed more thoroughly
in Section~5.2.1.

\subsection{Discussion}
\label{sec:\secname:discussion}
In general, the detection of transiting exoplanets with LSST will rely on
a small subset of potentially detectable planets that can be sufficiently
separated from statistical noise, rather than a clear threshold in a planet's
properties that would distinguish detectable planets vs. nondetectable planets.
This will mean that the best calculation of planet yields will have to come
from simulations of light curves for large numbers of stellar systems in order
to characterize LSST. The computation time involved in this process is sufficiently
prohibitive to prevent a metric being developed based directly on these
simulated light curves, however future work may be able to map relationships
between metric values for individual fields and the corresponding numbers
of planets that can be detected.

%
% % --------------------------------------------------------------------
%
% \subsection{OpSim Analysis}
% \label{sec:\secname:analysis}
%
% % --------------------------------------------------------------------
%
% \subsection{Discussion}
% \label{sec:\secname:discussion}
%
% ====================================================================

\navigationbar
