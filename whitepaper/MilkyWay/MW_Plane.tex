% ====================================================================
%+
% SECTION:
%    MW_Plane.tex
%
% CHAPTER:
%    .tex  % eg cosmology.tex
%
% ELEVATOR PITCH:
%    
% The Plane should be part of DWF
% It's possible some of this material should be moved to other sections, but it's here for now.
% 
% COMMENTS:
%
% AUTHOR:
%    Jay Strader (@caprastro)
%    Chris Britt
% ====================================================================

\section{The Galactic Plane Should Be Part of the Main Deep-Wide-Fast Survey}
\def\secname{MW_Plane}\label{sec:\secname} % For example, replace "keyword" with "lenstimedelays"

\noindent{\it Jay Strader, Chris Britt} % (Writing team)


The current baseline cadence (${\tt enigma\_1189}$) partially excludes the Galactic Plane from the deep-wide-fast survey and instead adopts a nominal 30 visits 
per filter as part of a special proposal. The basic justification for this cadence is that the crowding and extinction in the plane makes it impossible to 
achieve many of the extragalactic science goals of the survey. However, much Milky Way science can \emph{only} be accomplished in the plane, so it is not 
prima facie obvious that this area should be excluded from the deep-wide-fast survey. Careful consideration of science goals and metrics is necessary to 
determine whether the plane should be addressed in a special proposal---and the details of that proposal. Here we briefly highlight the science cases that 
focus on the Plane and show that despite the challenges, there is compelling support for including the Galactic Plane in the deep-wide-fast survey.

The science cases emphasized below do not require independent single epoch or combined photometry down to the nominal survey magnitude limits. However, it is 
still desirable to determine quantitative measurements of the effects of crowding and extinction on photometry, proper motions, and parallax as a function of 
Galactic latitude and filter rather than rough qualitative estimates. This will require image simulations through ImSim at a range of representative Galactic 
latitudes that are processed through the stack.

\subsection{Unique Galactic Transients and Variables}

Given the challenges of observing in the Plane, a simple increase of the survey area is insufficient justification for standard deep-wide-fast coverage. 
However, what is relevant is that many classes of important transients and variable stars are mostly or exclusively found in this region. Here we describe a 
representative (but non-exhaustive) list of these variables and transients.


\subsubsection{Stellar-Mass Black Holes}

Of the millions of stellar-mass black holes formed through the collapse of massive stars over the lifetime of the Milky Way, only $\sim 20$ have been 
dynamically confirmed through spectroscopic measurements (e.g., Corral-Santana et al.~2015). Many questions central to modern astrophysics can only be answered by enlarging this sample: which 
stars produce neutron stars and which black holes; whether there is a true gap in mass between neutron stars and black holes; whether supernova explosions 
result in large black hole kicks.

All of these known stellar-mass black holes were discovered  as bright X-ray sources, either due to persistent emission or as transient ``X-ray novae" 
due to an increase in the mass accretion rate within the standard  disk instability model. This introduces a strong bias in the known black hole candidates:
only those with short outburst recurrence times and whose peak X-ray luminosity reaches $\sim 10^{35}$--$10^{36}$ erg/s can be detected. 
Short-period binaries should be preferentially faint (Knevitt et al.~2014) and may be mostly or entirely missed in current samples, as evidenced by the current orbital period distribution of black hole binaries. While the historical rate of these detected outbursts is $\sim 1$ per year, the number of systems detected by LSST may be much higher since the majority of systems should be short period and X-ray faint, and hence undetected by X-ray all sky monitors. Careful simulations are required to robustly estimate the rate of new black hole transients which LSST could detect.
These outbursts are optically bright, from $4-8$ magnitudes reaching $M_{V}\sim 0$, and so should be easily visible at large distances even near the Plane. Some of the optical light is contributed by the jet, which is quenched at the peak of the outburst, returning later on as the source fades (Jain et al. 2001). 

More importantly, there is expected to be a large population of black hole binaries in quiescence with low X-ray luminosities from $\sim 10^{30}$--$10^{33}$ erg/s.
Such systems can be identified as optical variables that show unique, double-humped ellipsoidal variations of typical amplitude $\sim 0.2$ mag due to the tidal deformation of the secondary star, which can be a giant or main sequence star. In some cases analysis of the light curve alone can point to a high mass ratio between the components, suggesting a black hole primary;
in other cases the accretion disk will make a large contribution to the optical light which results in intrinsic, random, and fast variations in the light curve. The disk contribution to optical light can change over time, and several years of data is necessary to properly subtract the accretion disk contribution in order to properly fit ellipsoidal veriations (Cantrell et al. 2010). These variations could provide an additional way to distinguish them from contact binaries, and simulations to determine what sampling is necessary to distinguish them in this way is needed. Additional follow up on objects discovered by LSST would be required to robustly identify the nature of such an accreting binary. Some of the metrics used in \S 6.5 will be of use here as well. 

Most black hole candidates have been identified toward the center of the Galaxy, in the Plane: 68\% of black hole candidates are located within $5^{\circ}$ of the Plane and 92\% within $10^{\circ}$ of the Plane. Therefore, including the Plane in the deep-wide-fast survey offers the best route to identifying the hundreds of quiescent black hole binaries that should be present. The brighter sources will be amenable to spectroscopy with the current generation of 4-m to 10-m telescopes to dynamically confirm new black holes; spectroscopy of all candidates should be possible with the forthcoming generation of large telescopes.

\subsubsection{Other Compact Binaries}

The methods discussed above for black holes can also be used to identify white dwarf and especially neutron star binaries, which again are preferentially found in the dense regions of the Galaxy.
In some cases the secondaries can show clear ellipsoidal variations and be identified and studied as periodic variables; in other cases the accretion disk light can dominate, resulting in quasi-periodic or
non-periodic light curves. Nonetheless, cross-matching LSST variables against the all-sky eROSITA catalog (reaching typical X-ray fluxes of $10^{31}$--$10^{32}$ erg/s at 8 kpc) will reveal hundreds of new neutron star X-ray binaries (Merloni et al.~2012). 

From studies of \emph{Fermi} $\gamma$-ray sources, it is clear that pulsar surveys miss a huge fraction of pulsars in close binary systems (probably due to the effects of enshrouding material) but that targeted follow-up can detect such pulsars (Ray et al.~2012). Combining these observations with optical spectroscopic data gives the equivalent of a double-lined spectroscopic binary, and can often yield accurate masses for both the neutron star and the secondary. Further, gold-standard Shapiro-delay masses for neutron stars require edge-on orientations that only arise
from large samples of objects. Such studies offer the best chance to probe the upper end of the neutron star mass distribution; the maximum mass of neutron stars determines the equation of state of dense matter and is of central interest to both astronomers and nuclear physicists. Additionally, proper motions of neutron star X-ray binaries should be easily detectable; dynamical kicks from supernovae, though poorly understood, are on the order of hundreds km\,s$^{-1}$. It is unclear what, if any, dynamical kicks black holes may undergo at birth. A thorough understanding of the true distribution of dynamical kicks from supernovae will place strong observational constraints on the internal physics of Type II supernovae. 

A great many dwarf novae from white dwarf binaries will be discovered routinely by LSST, preferentially towards low Galactic latitudes. Dwarf novae can be standardizable candles (Patterson et al. 2011) though they tend to be fainter than stars on the instability strip, even in outburst ($M_{V}\sim 3-6$). The total number of dwarf novae is, however, poorly understood---theoretical estimates routinely yield a significantly higher number than are observed in the solar neighborhood. The selection biases and low number density make this comparison difficult, but understanding the true specific frequency of these systems provides a key check on common envelope evolution, which is poorly understood and has a large impact on, for example, LIGO event rates. LSST will detect dwarf novae, which last at least several days, out to kpc scales. This will allow a test of not only the number of cataclysmic variables, but also of the 3D distribution within the Galaxy and dependence on metallicity gradients (Britt et al. 2015). The cadence of observations is critical in obtaining an accurate measure of the population of cataclysmic variables, as a long baseline is necessary to recover low duty cycle systems while observations spaced too far apart risk leaving windows large enough to allow short outbursts to go undetected. Appropriate simulations of what science returns is allowed by various cadence schedules are critical.

\subsubsection{A Milky Way Supernova}

A supernova in the Milky Way would be among the most important astronomical events of our lifetime, with enormous impacts on stellar astrophysics, compact 
objects, nucleosynthesis, and neutrino and gravitational wave astronomy. The estimated rate of supernovae (both core-collapse and Type Ia) in the Milky Way 
is about 1 per 20--25 years (Adams et al.~2003); hence there is a 40--50\% chance that this would occur during the 10-year LSST survey.

If fortunate, such an event will be located relatively close to the Sun and will be an easily observed (perhaps even naked-eye) event. However, we must be 
cognizant of the likelyhood that the supernova could go off in the mid-Plane close to the Galactic Center or on the other side of the Milky Way---both 
regions covered by LSST. While any core-collapse event will produce a substantial neutrino flux, alerting us to its existence, such observations will not 
offer precise spatial localization. The models of Adams et al.~(2013) indicate that LSST is the \emph{only} planned facility that can offer an optical 
transient alert of nearly all Galactic supernovae. Even if the supernova is not too faint, LSST will likely be the sole facility with synoptic observations 
preceding the explosion, providing essential photometric data leading up to the event---but only if LSST covers the Plane at a frequent cadence.


\subsubsection{Novae}

Only $\sim 15$ novae (explosions on the surfaces of white dwarfs) are discovered in the Milky Way each year, while observations of external galaxies show 
that the rate should be a factor of $\sim 3$ higher (Shafter et al.~2014). Evidently, we are missing 50--75\% of novae due to their location in crowded, 
extinguished regions, where they are not bright enough to be discovered at the magnitude limits of existing transient surveys. Fundamental facts about novae 
are unknown: how much mass is ejected in typical explosions; whether white dwarfs undergoing novae typically gain or lose mass; whether the binary companion 
is important in shaping the observed properties of nova explosions. Novae can serve as scaled-down models of supernova explosions that can be tested in 
detail, e.g., in the interaction of the explosion with circumstellar material (e.g., Chomiuk et al.~2015). Further, since accreting white dwarfs are prime 
candidates as progenitors of Type Ia supernovae, only detailed study of novae can reveal whether particular systems are increasing toward the Chandrasekhar 
mass as necessary in this scenario.

Most novae occur in the Galactic Plane and Bulge, and therefore the quick identification of novae is best enabled by the standard deep-wide-fast cadence. 
These events will trigger multi-wavelength follow-up ranging from the radio to X-ray and $\gamma$-rays; these data are necessary for accurate measurements of 
the ejected mass.

\subsection{Microlensing}

Gould (2013) shows that coverage of the Plane with the standard deep-wide-fast cadence would be, in essence, an effective intra-disk microlensing survey (in 
which disk stars are lensed by other objects in the disk, such as exoplanets, brown dwarfs, or compact objects). The lower stellar density compared to past 
bulge-focused microlensing surveys would be offset by the larger area covered by LSST. The predicted rate of high magnification microlensing events that are 
very sensitive to planets would be $\sim 25$ per year. This survey would be able to detect planets at moderate distances from their host stars, a regime 
poorly probed by standard Doppler and transit techniques. The LSST data alone would not be sufficient: the detection of a slow ($\sim$ days) timescale 
increase in brightness of a disk star would need to trigger intensive photometric observations from small (1-m to 2-m class) telescopes that would observe at 
high cadence for the 1--2 months of the microlensing event. This science \emph{requires} the standard deep-wide-fast cadence (with typical inter-night visit 
times of a few days) to catch lensing events as they start to brighten.


\subsection{A Three-Dimensional Dust Map of the Milky Way}

The Pan-STARRS1 survey (PS1) has produced a three-dimensional dust map of the region of the sky covered in their 3$\pi$ survey (which excludes a large part 
of the Galactic Plane toward the south). Such maps are necessary to accurately measure the intrinsic luminosities and colors of both Galactic and 
extragalactic sources. The PS1 map (Schlafly et al.~2014) saturates at extinctions $E(B-V) > 1.5$ as their tracer stars fall out of the survey catalogs 
fainter than $g\sim 22$, meaning that this high-fidelity map does not extend uniformly to within a few degrees of the midplane. In addition, it only extends 
to a distance of about 4.5 kpc. Deep LSST data will allow this map to be extended to much higher extinctions and larger distances. Owing to the high 
extinction and the use of blue filters, this project is less affected by crowding than other projects requiring photometry in the Plane. Nonetheless, 
quantiative estimates of the expected photometric accuracy in coadded $u$ and $g$ images at low Galactic latitude are desirable.


% --------------------------------------------------------------------

% \subsection{Target measurements and discoveries}
% \label{sec:\secname:targets}

% Describe the discoveries and measurements you want to make.

% Now, describe their response to the observing strategy. Qualitatively,
% how will the science project be affected by the observing schedule and
% conditions? In broad terms, how would we expect the observing strategy
% to be optimized for this science?


% --------------------------------------------------------------------
%
 \subsection{Metrics}
 \label{sec:\secname:metrics}
%

Using proposed variable star metrics (\S 6.3), the recovery of accurate periods for variables discussed above should be compared between the baseline cadence and the
results of the ``Normal Plane" OpSim proposal (in which the portion of the Plane falling within the deep-wide-fast declination limits is observed at the
standard deep-wide-fast cadence).

Several focused surveys on the Bulge with the wide-field MOSAIC II and DECam instruments have demonstrated that \emph{differential} photometry is possible in
these very crowded regions down to close to the LSST single epoch photometry limits in $r$ and $i$ (e.g., Britt et al.~2014). Therefore, periods and
amplitudes for variables can be accurately measured even if precise absolute photometry is not available for every star (even though such photometry should
be possible down to at least the main sequence turnoff; Rich 2015). In the Plane, away from the Bulge, the crowding is less extreme, and quantitative limits
should be calculated for photometry, proper motion, and parallax measurements through image simulations as noted above.

%
%
% --------------------------------------------------------------------

%\subsection{OpSim Analysis}
%\label{sec:\secname:analysis}


% --------------------------------------------------------------------

% \subsection{Discussion}
% \label{sec:\secname:discussion}
%
% Discussion: what risks have been identified? What suggestions could be
% made to improve this science project's figure of merit, and mitigate
% the identified risks?
%
%
% ====================================================================

\navigationbar
