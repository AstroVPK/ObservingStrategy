% ====================================================================
%+
% SECTION:
%    section-name.tex  % eg lenstimedelays.tex
%
% CHAPTER:
%    chapter.tex  % eg cosmology.tex
%
% ELEVATOR PITCH:
%    Explain in a few sentences what the relevant discovery or
%    measurement is going to be discussed, and what will be important
%    about it. This is for the browsing reader to get a quick feel
%    for what this section is about.
%
% COMMENTS:
%
%
% BUGS:
%
%
% AUTHORS:
%    Phil Marshall (@drphilmarshall)  - put your name and GitHub username here!
%-
% ====================================================================

\section{Mapping the Milky Way Halo}
\def\secname{MW_Halo}\label{sec:\secname} % For example, replace "keyword" with "lenstimedelays"

\noindent{\it Kathy Vivas, Colin Slater, David Nidever}  % (Writing team)

The study of the Halo of the Milky Way is of the most importance not only to understand
the formation and early evolution of our own galaxy, but also to test 
test current models of hierarchical galaxy formation. 
LSST will provide an unprecedented combination of
area, depth, multi-band, multi-epoch information for pursuing detail studies
of the structure of this old Galactic component. We focus here in three
specific projects that can be pursued with LSST. We define metrics that can
be calculated in order to quantify the feasibility of the projects under different
observational strategies. We expect more projects will join later.

RR Lyrae stars have been known for several decades as excellent tracers
of the halo population. They are not only old stars ($>10$ Gyrs) but they are
also excellent standard candles that allow to build 3-dimensional maps. 
The halo of the Milky Way has been now surveyed in a very large extension up to 
$\sim 60-80$ kpc from the Galactic center (refs). Beyond that, the halo is
mostly uncharted territory.
From these RR Lyrae surveys, we have learned that the halo is filled with substructures
which are usually interpreted as debris from destroyed satellite galaxies. The smooth 
component of the RR Lyrae distribution is well described
with a power-law of the mean number density of RR Lyrae stars as a function of
galactocentric distance, which gets steeper after $\sim 30$ kpc (refs). 
Thus, beyond $\sim 60$ kpc, few field RR Lyrae stars are expected. However, we expect that 
any RR Lyrae star beyond this distance may be part of either debris material or distant
satellite galaxies of low luminosity that have been escaped detection until now (refs). 
LCDM models predict debris as far as XXX kpc from the galactic center
This is the territory that will be explored by LSST.

Similarly, red giant stars can be used to trace the structure of the halo up to large
distances. They have the advantage 
of being bright and numerous stars. COLIN, PLEASE STEP IN HERE.

Fainter than these two tracers, main sequence stars stand up as a tool for studying
the Halo. They are the most numerous type of stars available and statistical studies 
are possible. Using the technique of photometric metallicities (Ivezic et al 2008), 
SDSS provided unprecedented maps of the metallicity distribution up to  $\sim 10$ 
kpc from the Galactic center, unveiling not only the mean metallicity distribution 
of the halo but also, sub-structures within the halo. This kind of works will be extended
to the outermost parts of the Galaxy with LSST data.


% --------------------------------------------------------------------

\subsection{Target measurements and discoveries}
\label{sec:keyword:MW_Halo_targets}

The three projects just described require the discovery and/or measurement of the following 
type of objects:

\begin{itemize}

\item RR Lyrae stars: These are bright horizontal-branch variable stars with
periods between 0.2 to 1.0 days and large amplitudes, particularly in the bluer 
bandpasses (g amplitudes = XX to XX). Optimal use of LSST for discovering
RR Lyraes involves the simultaneous use of multi-band time series (refs).
Chapter XX discusses with details the discovery metrics for RR Lyrae stars.
A particularly valuable measurement for studies in the halo is the infrared mean
magnitudes z and Y since they provide the most accurate way to obtain 
distances.

\item Main sequence stars: lacking any distinguishable variability, the
challenge in selecting a large and clean sample of main sequence stars comes
from tremendous number of small and nearly-unresolved galaxies present at
faint magnitudes. Precise star/galaxy separation is thus the limiting factor
on the useful depth of the main sequence sample. In addition to identifying
dwarfs, using dwarfs to map the metallicity distribution of the halo requires
precise u-band data, since it exhibits the strongest metallicity dependence of
the LSST filters.

\item Red Giants: due to their intrinsic luminosity red giants will be sample
a far deeper volume than main sequence stars at similar apparent magnitudes,
but they must first be identified and separated from the very numerous main
sequence stars present in the field. A gravity-sensitive photometric index can
be used for separating efficiently giants from dwarfs (refs). The u magnitude
is an essential ingredient in this process and it is necessary to follow-up
the behavior of the u limiting magnitude under different observational
strategies.

\end{itemize}

% --------------------------------------------------------------------

\subsection{Metrics}
\label{sec:keyword:MW_Halo_metrics}

\textbf{Star-Galaxy Separation:} For main sequence stars, the useful depth of
the survey will likely not be the photometric detection limit but will instead
be set by the ability to differentiate stars from unresolved background
galaxies. Towards faint magnitudes the contamination by galaxies worsens
significantly for several reasons: the number of galaxies is rising
substantially, the angular size of galaxies is shrinking, and our ability to
distinguish stars from marginally resolved galaxies diminishes for faint
sources simply due to photon statistics. While the fundamental properties of
the contaminant sources is beyond our control, our ability to reject these
sources depends on survey parameters such as the distribution of seeing across
visits and the depth of these visits.

Our star galaxy separation metric accounts for these factors, using a modeling
framework described in (s/g paper ref). This model uses the distribution of
galaxies in size and number, derived from HST COSMOS observations, along with
a fully Bayesian model decision formalism to compute the expected completeness
and contamination in star-galaxy separation. Computationally for each position
in the survey footprint we interpolate the results from that work on a grid in
seeing, galaxy size, and coadd depth, then integrate over the distribution of
galaxy sizes.

[… will have more to say about the actual metric once it is fully implemented]


% --------------------------------------------------------------------

\subsection{OpSim Analysis}
\label{sec:keyword:MW_Halo_analysis}

\begin{itemize}

\item Comment on the north ecliptic spur in enigma\_1189. Is it close to the
WFD S/G limit?

\item Pan-STARRS-like cadence ops2\_1092, observing up to dec +15. How much
volume do we gain, and how much do we lose in the WFD survey?

\end{itemize}


% --------------------------------------------------------------------

\subsection{Discussion}
\label{sec:keyword:MW_Halo_discussion}

Discussion: what risks have been identified? What suggestions could be
made to improve this science project's figure of merit, and mitigate
the identified risks?


% ====================================================================

\navigationbar
