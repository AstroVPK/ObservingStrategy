%\chapter{Mapping Our Galaxy: Positions, Proper Motions and Parallax}
%\def\chpname{astrometry}\label{chp:\chpname}
\section{Astrometry with LSST: Positions, Proper Motions, and Parallax}
\def\secname{MW_Astrometry}\label{sec:\secname} % For example, replace "keyword" with "lenstimedelays"

\noindent {\it Dave Monet, Dana Casetti, John Gizis, Michael Liu}

\subsection{Introduction: Astrometry as a special case}
\label{sec:keyword:MW_Astrometry_intro}

A number of Milky Way science cases of interest to the Astronomical
community will depend critically on the astrometric accuracy LSST will
deliver. While ``astrometry'' is not a science case in the framework
of this Whitepaper, LSST's astrometric performance will be sensitive
to the particular choice of observing strategy.
%While astrometry is not a science case, high astrometric accuracy enables
%a large number of science cases.  
Hence, the LSST Observing Strategy needs to be examined for systematic
trends that might mitigate or even preclude precise measures of
stellar positions, proper motions, parallaxes, and perturbations that
arise from unseen companions. To illustrate the broad impact of
strategy-enabled astrometry, three example impacted science cases are:
%that illustrate the various impacts of the observing strategy might
%be:
%To highlight the
%various astrometric impacts of the strategy, three science cases have
%been chosen for particular attention:
\begin{itemize}
\item The tie between the Radio and Optical realizations of the
International Celestial Reference System.
\item Identification of Streams in the Galactic Halo using proper motions.
\item The specific and ensemble agreement between LSST and Gaia parallaxes.
\end{itemize}
Each of these cases stresses different aspects of the LSST hardware, software,
and observing strategies.

\subsection{Sensitivity of parallax measurements to observing strategy}
\label{sec:keyword:MW_Astrometry_cadence}

%\medskip
The measurement of stellar parallax puts the most constraints on the
observing cadence.  There are two major issues:
\begin{itemize}
\item Sampling over a wide range of parallax factor.
\item Breaking the correlation between Differential Color Refraction
and parallax factor.
\end{itemize}
The parallax factors characterize the ellipse of the star's apparent motion
as seen during the year.  The shape of the ellipse is given by the Earth's
orbit and is not a free parameter in the astrometric solution.  The
amplitude of the RA parallax factor is close to unity while the amplitude
of the Dec parallax factor is dominated by the sine of ecliptic latitude.
The RA parallax factor has maximum amplitude when the star is approximately
six hours from the Sun, so the optimum time for parallax observing is when the
star is on the meridian near evening or morning twilight.
Atmospheric refraction displaces the star's apparent position in the
direction of the zenith by an amount characterized by both the wavelength
of the light and the distance to the zenith.  Whereas the measured position
of star is a function of the total refraction, the measurement of parallax
and proper motion depends on the differences in the refraction as a function
of the color of each star and the circumstances of the observations.  This
dependence is called Differential Color Refraction (DCR).
The combination of parallax factor and DCR leads to the these two rules
well known to those who make astrometric observations.
\begin{itemize}
\item [1] Observations need to cover the widest possible range in parallax
factor.
\item [2] The correlation between parallax factor and hour angle in the
observations needs to be minimized.
\end{itemize}

The search for faint proper motion stars has two key components.  The first
is the need to identify stars that move from the ensemble of other image
features that can cause confusion.  For example, a compact group of stars
that contains one or more stars of variable brightness can confuse the catalog
correlation algorithm.  The other is the need to establish the zero point.
For the case of relative astrometry, meaning the measurement of relative
positions in an image, the question remains on how to remove the mean motion
of the reference frame.  For example, astrometry on certain classes of galaxies
might produce a zero point of sufficient accuracy.  This leads to a third
constraint on the observing cadence.
\begin{itemize}
\item [3)] Observations must cover a sufficient range of epochs so that stars with
linear or periodic motions can be identified at a high level of confidence.
\end{itemize}

The tie between the radio and optical reference frames relies on measuring
accurate positions for objects visible in both wavelength regimes.  Whereas
there are optical variable stars with radio emission, most have associated
optical nebulosity that degrades the accuracy of the optical positions.
The typical radio+optical object is a QSO.  Unfortunately, many QSOs have
detectable optical or radio structures that degrade the positions or
suggests a displacement between the location of the sources of the radio
and optical radiation.  The major contribution from LSST will be the
identification of a large number of QSOs based on their colors that have
minimal (if any) spatially extended structure.  The impact of this search
has no obvious impact on the cadence other than temporal coverage to
identify variability.

\subsection{Metrics for LSST's delivered astrometric accuracy}
\label{sec:keyword:MW_Astrometry_metrics}

%\medskip
In summary, there are three metrics for the observing strategy that have direct
relevance
on the quality of LSST astrometric measurements.  These were identified years
ago and are already in the suite of MAF utilities, but they should be
reviewed prior to making final decisions.

\begin{itemize}
\item[A)] For each LSST field, the parallax factors at each epoch of
observation need to be computed.  The ensemble of these must be checked for
sufficient coverage of the parallactic ellipse.  In particular, the number of
measures with RA parallax factor less than -0.5 and greater than +0.5
needs to be tallied because these carry the most weight in the solution
for the amplitude (parallax).
\item[B)] For each LSST field, the hour angle of the observation needs to be
computed, and the correlation between hour angle and parallax factor
needs to be examined for significance.  The observing strategy must minimize
the number of fields with this correlation.
\item[C)] The epochs of observation for each field must be checked for a
reasonable coverage over the duration of the survey and to avoid
collections of too many visits during a few short intervals.
\end{itemize}

\subsection{Topics that will need to be addressed}
\label{sec:keyword:MW_Astrometry_furtherwork}

%\medskip
Finally, it must be noted that these MAF metrics are only part of the
study of LSST's predicted astrometric performance.  Detailed simulations
and studies need to be done in many other areas as part of the
prediction and verification of LSST's astrometric performance.  Among
the most important are the following.

\begin{itemize}
\item Can we use galaxies as reference objects, and if so are certain
shapes or colors better than others?
\item Can we identify QSOs and sense optical structure that might
mitigate using certain ones in the Radio-Optical reference frame link?
\item Given the LSST exposure time, site, and physical characteristics,
how can we mitigate the limitations on astrometric accuracy imposed
by the seeing and local atmospheric turbulence?
\item At what star densities does the measurement of a centroid become
difficult or impossible, and does difference imaging allow us to work
in these crowded areas?
\item What tools do we need to compare the general and specific agreement
between the {\it Gaia} results and the LSST results?
\item Does the ``brighter-wider" effect in the deep depletion CCDs introduce
a magnitude term into the centroid positions?
\end{itemize}

\navigationbar

% --------------------------------------------------------------------
