%\chapter{Mapping Our Galaxy: Positions, Proper Motions and Parallax}
%\def\chpname{astrometry}\label{chp:\chpname}
\section{Astrometry with LSST: Positions, Proper Motions, and Parallax}
\def\secname{MW_Astrometry}\label{sec:\secname} % For example, replace "keyword" with "lenstimedelays"

\noindent {\it Dave Monet, Dana Casetti, John Gizis, Michael Liu, Jay Strader}

A number of Milky Way science cases of interest to the Astronomical
community will depend critically on the astrometric accuracy LSST will
deliver. While ``astrometry'' is not a science case in the framework
of this white paper, LSST's astrometric performance will be sensitive
to the particular choice of observing strategy.
%While astrometry is not a science case, high astrometric accuracy enables
%a large number of science cases.  
Hence, the LSST Observing Strategy needs to be examined for systematic
trends that might mitigate or even preclude precise measures of
stellar positions, proper motions, parallaxes, and perturbations that
arise from unseen companions. Here we highlight two representative science cases
that illustrate the broad impact of strategy-enabled astrometry.


%Each of these cases stresses different aspects of the LSST hardware, software,and observing strategies.

%, here we highlight three representative science cases.
%that illustrate the various impacts of the observing strategy might
%be:
%To highlight the
%various astrometric impacts of the strategy, three science cases have
%been chosen for particular attention:

%\subsection{Introduction: Astrometry as a special case}
%\label{sec:keyword:MW_Astrometry_intro}

\subsection{Target Measurements and Discoveries}

\begin{itemize}
\item Identification of Streams in the Galactic Halo using proper motions.
\item A complete sample of stars in the solar neighborhood.
\end{itemize}
%\item The tie between the Radio and Optical realizations of the International Celestial Reference System.
%\item The specific and ensemble agreement between LSST and Gaia parallaxes.


{\bf 1. Streams in the Galactic halo}

Much of the Milky Way's stellar halo was built by the accretion of smaller galaxies. Given that these galaxies
were generally of low mass, their tidal debris should still form coherent structures in phase space, especially
in the outer Galaxy where dynamical times are long. The identification of these streams would allow
a reconstruction of the accretion history of the Milky Way. Tides also lead to the dissolution of globular clusters,
leaving notably thin streams that serve as sensitive tracers both of the Galactic potential and of the presence of dark
subhalos.

A relatively small number of streams, originating from both dwarfs and globular clusters, have been identified via photometry
of individual stars in large surveys such as SDSS. However, only the highest surface brightness structures can be found
in this manner, and it is often difficult to trace the streams over their full extent. LSST will enable streams to be identified 
by stellar proper motions, and combined with targeted follow-up spectroscopy, will yield full 6-D position and velocity measurements suitable for dynamical modeling.
Further, it will allow the discovery of tidal debris that is no longer spatially coherent but which can be unambiguously identified in phase space.

{\it Response to observing strategy:} Most stars in streams will be main-sequence stars, and the old main sequence turnoff  is located at $r\sim24$ at a distance of 100 kpc.
The nominal LSST proper motion precision at this magnitude is 1 mas yr$^{-1}$, corresponding to about 475 km s$^{-1}$ at this distance. The proper motion
measurements will be better for brighter stars, but in general ensembles of stars will be necessary for accurate measurements. To make accurate proper motion measurements for faint stars, several key components are required. First, a zero point must be established, possibly via background galaxies located in each field. Next, the observations must cover a sufficient range of epochs to reliably detect linear proper motions.

To identify streams over their full lengths of many degrees of the sky, relative astrometry over small fields will not be sufficient. Therefore the absolute astrometric frame is important. Matching the optical astrometry to the radio International Celestial Reference System (ICRS) relies on measuring accurate positions for objects visible in both wavelength regimes.
These are typically distant QSOs. Unfortunately, many QSOs have detectable optical or radio structures that degrade the positions or suggests a displacement between the location of the sources of the radio and optical radiation. LSST will need to identify a large number of point-like QSOs based on their colors and variability.

%The tie between the radio and optical reference frames relies on measuring accurate positions for objects visible in both wavelength regimes.  Whereas there are optical variable stars with radio emission, most have associated optical nebulosity that degrades the accuracy of the optical positions. The typical radio+optical object is a QSO.  Unfortunately, many QSOs have detectable optical or radio structures that degrade the positions or suggests a displacement between the location of the sources of the radio and optical radiation.  The major contribution from LSST will be the identification of a large number of QSOs based on their colors that have minimal (if any) spatially extended structure.  The impact of this search has no obvious impact on the cadence other than temporal coverage to identify variability.

{\bf 2. A complete sample of stars in the solar neighborhood}

The direct solar neighborhood offers our only chance to get make a complete sample of stars, brown dwarfs, and stellar remnants that encompass the entire formation and dynamical history of the Milky Way. While Gaia will offer parallax measurements for perhaps billions of stars, its faint magnitude limit of $G\sim 20$ will limit its measurements of the lowest-mass objects
and remnants to nearby objects, much less than the thin disk scale height of $\sim 300$ pc. For example, Gaia can only measure parallaxes for $0.2 M_{\odot}$ M dwarfs to about 100 pc
and $0.1 M_{\odot}$ M dwarfs to only \emph{10 pc}, showing that Gaia is ill-suited for studies of the coolest dwarfs. By contrast, LSST can measure parallaxes for $> 10^5$ M dwarfs and thousands of L/T brown dwarfs (the coolest Y dwarfs are too faint even for LSST; little contribution is likely here beyond the sample provided by WISE). Gaia will likewise be limited to cool white dwarfs within $\sim 100$ pc with which to estimate the age of the disk, and the thick disk and halo will be out of reach. LSST can directly compare white dwarf luminosity functions to determine precise differential ages for the thin disk, thick disk, and halo.

{\it Response to observing strategy:} Successfully completing this project will require parallax measurements much fainter than possible with Gaia as well as a verification that the LSST and Gaia parallax measurements are consistent in the overlapping magnitude range.

The measurement of stellar parallax puts the substantial constraints on the observing cadence. There are two major issues: the need to sample a wide range of parallax factor (related to time of year), and breaking the correlation between differential color refraction and parallax factor.

``Parallax factors" characterize the ellipse of the star's apparent motion as seen over the course of a year. The shape of the ellipse is given by the Earth's orbit and is not a free parameter in the astrometric solution. The amplitude of the right ascension parallax factor is close to unity while the amplitude of the declination parallax factor is dominated by the sine of ecliptic latitude.
The right ascension parallax factor has maximum amplitude when the star is approximately six hours from the Sun, so the optimum time for parallax observing is when the
star is on the meridian near evening or morning twilight. Atmospheric refraction displaces the star's apparent position in the direction of the zenith by an amount dependent on both the wavelength of the light and the distance to the zenith. Whereas the measured position of star is a function of the total refraction, the measurement of parallax
and proper motion depends on the differences in the refraction as a function of the color of each star and the circumstances of the observations.  This
dependence is called differential color refraction. The combination of parallax factor and differential color refraction leads to two rules:
\begin{itemize}
\item [1] Observations need to cover the widest possible range in parallax
factor.
\item [2] The correlation between parallax factor and hour angle in the observations needs to be minimized.
\end{itemize}

%with respect to the meanmotion of the reference frame. 

%The search for faint proper motion stars has two key components.  The first is the need to identify stars that move from the ensemble of other image features that can cause confusion.  For example, a compact group of stars that contains one or more stars of variable brightness can confuse the catalog correlation algorithm.  The other is the need to establish the zero point. For the case of relative astrometry, meaning the measurement of relative positions in an image, the question remains on how to remove the mean motion of the reference frame.  For example, astrometry on certain classes of galaxies might produce a zero point of sufficient accuracy.  This leads to a third constraint on the observing cadence.
% \begin{itemize}
%\item [3)] Observations must cover a sufficient range of epochs so that stars with
%linear or periodic motions can be identified at a high level of confidence.
%\end{itemize}


%\subsection{Sensitivity of parallax measurements to observing strategy}
%\label{sec:keyword:MW_Astrometry_cadence}

%\medskip


\subsection{Metrics for LSST's delivered astrometric accuracy}
\label{sec:keyword:MW_Astrometry_metrics}

%\medskip

First we discuss metrics for the observing strategy that affect all of LSST's astrometric
measurements, then discuss those relevant for the stream science case.

These general metrics were identified years ago and are already in the suite of MAF utilities, but they should be
reviewed prior to making final decisions.

\begin{itemize}
\item[A)] For each LSST field, the parallax factors at each epoch of
observation need to be computed.  The ensemble of these must be checked for
sufficient coverage of the parallactic ellipse.  In particular, the number of
measures with RA parallax factor less than --0.5 and greater than +0.5
needs to be tallied because these carry the most weight in the solution
for the amplitude (parallax).
\item[B)] For each LSST field, the hour angle of the observation needs to be
computed, and the correlation between hour angle and parallax factor
needs to be examined for significance.  The observing strategy must minimize
the number of fields with this correlation.
\item[C)] The epochs of observation for each field must be checked for a
reasonable coverage over the duration of the survey and to avoid
collections of too many visits during a few short intervals.
\end{itemize}

For the stream project discussed above, a simple to state (but perhaps complex to implement) metric
is the number of streams that can be discovered in LSST via their proper motions. As a first
attempt, it would be reasonable to assume about 100 halo streams from old, metal-poor dwarf galaxies with
stellar masses $10^5-10^7 M_{\odot}$ distributed as $r^{-3.5}$. The stream widths and internal velocity
dispersions can be set from galaxy scaling relations, and their 3-D velocities consistent with a simple Galactic mass
model at their radii. Setting the stream lengths is more complicated, but should cover a large range from a few to many kpc.
Over a given area, the stream ``S/N" can roughly be taken as the number of stream stars (identified via proper motion, color, and magnitude)
divided by the square root of the number of field stars. For globular clusters, a similar number of streams could be included, but these should have much smaller widths (10s of pc)
and typical masses $10^4-10^5 M_{\odot}$. Eventually it would be desirable to use actual simulated stream parameters taken from cosmological models of the Milky Way (e.g.,
from the Aquarius simulation).

The  solar neighborhood project will be sensitive to the general parallax and proper motion metrics discussed above. However, more specific metrics
can also be useful: for example, the precision of the differential age measurement between the thin disk and halo, which would depend on the number of white dwarfs that can be isolated
from each population.

\subsection{Topics that will need to be addressed}
\label{sec:keyword:MW_Astrometry_furtherwork}

%\medskip
Finally, it must be noted that these MAF metrics are only part of the
study of LSST's predicted astrometric performance.  Detailed simulations
and studies need to be done in many other areas as part of the
prediction and verification of LSST's astrometric performance.  Among
the most important are the following.
\begin{itemize}
\item How well do galaxies perform as astrometric reference objects? Are certain shapes or colors better than others? What is the 
surface density of ``good" astrometric reference galaxies as a function of filter? 
\item How well can we identify optically point-like QSOs that will be useful in matching the optical reference frame to the ICRS?
%\item Given the LSST exposure time, site, and physical characteristics, how can we mitigate the limitations on astrometric accuracy imposed by the seeing and local atmospheric turbulence?
\item How does the astrometric performance depend on stellar density? If there are fields in which photometry is only possible via difference imaging, what are the limitations
on astrometry in these fields?
%\item What tools do we need to compare the general and specific agreement between the {\it Gaia} results and the LSST results?
\item Does the ``brighter-wider" effect in the deep-depletion CCDs introduce a magnitude term into the centroid positions?
\end{itemize}

\navigationbar

% --------------------------------------------------------------------
