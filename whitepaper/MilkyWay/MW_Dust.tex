% ====================================================================
%+
% SECTION:
%    MW_Dust.tex
%
% CHAPTER:
%    galaxy.tex
%
% ELEVATOR PITCH:
%    Metric suggested - uncertainty and bias in $E(B-V)$~estimates as a
%    function of location on-sky - but not yet implemented.
%-
% ====================================================================

% \section{Dust in the Milky Way}
\subsection{Dust in the Milky Way}
\def\secname{MW_Dust}\label{sec:\secname}

\credit{pmmcgehee},
\credit{willclarkson}

As discussed in the LSST Science Book (particularly its Section 7.5),
possession of an accurate, three-dimensional dust map is important to
many astrophysical studies. The two most significant all-sky maps
generated in the past two decades are the SFD98 maps based on IRAS
observations, and the recent thermal dust maps derived from Planck
submillimeter data. The angular resolutions of both maps are similar -
between 4 to 6 arcminutes.

Both of the aforementioned maps are strictly two-dimensional and contain
no information about the distribution of dust along the line of sight. A
third dimension can be obtained by analysis of accurate stellar
photometry which constrains both the reddening $E(B-V)$ and extinction
$R_V \equiv A_V/E(B-V)$ towards individual stars. This approach requires
determination of the intrinsic stellar colors and the photometric
parallax of each star in the presence of an unknown amount and law of
extinction. Such maps are necessary to accurately measure the intrinsic
luminosities and colors of both Galactic and extragalactic sources.
Recent work on 3-D maps include the Bayesian analysis method based on
Pan-STARRS--1 (PS1) data \citep{green15} and an alternative technique
using SDSS photometry of M dwarfs (McGehee et al. 2016, in preparation).
However, these studies have so far typically been limited to
heliocentric distances of $\lesssim$4.5 kpc. In the full co-added
survey, LSST will be able to map dust structures out to distances
exceeding 40 kpc, thus revealing a detailed picture of this component of
the Milky Way Galaxy.

While the PS1 3-D dust map is a significant advance, it suggests a
number of major improvements that LSST will be able to
provide. Firstly, the PS1 map covers the region of the sky covered in
their 3$\pi$ survey, and thus excludes a large part of the Galactic
Plane towards the South. Secondly, the PS1 map \citep{2014ApJ...789...15S}
saturates at extinctions $E(B-V) > 1.5$ as their tracer stars fall out
of the survey catalogs fainter than $g\sim 22$, meaning that this
high-fidelity map does not extend uniformly to within a few degrees of
the midplane. Thirdly, this map is currently limited to distances $d
\lesssim 4.5$~kpc. Deep LSST data will allow this map to be extended
to much higher extinctions and larger distances. Owing to the high
extinction and the use of blue filters, this project is less affected
by crowding than other projects requiring photometry in the Plane.

The SDSS approach is complementary, and makes use of reddening-free
colors defined in the SDSS $ugriz$~system by \citet{mcgehee05}.
This approach makes use of M dwarf locus in $(g-r,r-i)$
being nearly perpendicular to the reddening vector in that color-color
space. This allows mapping of a reddening-invariant index to the
intrinsic stellar $g-i$ color and subsequent determination of the
light-of-sight reddening. This approach assumes a set extinction law,
i.e $R_V = 3.1$, in order compute the reddening-invariant index from
the observed $g-r$ and $r-i$ colors. Given the relative faintness of M
dwarfs, this technique is distance limited to $\sim$1 kpc when based
on SDSS data. With its significantly greater survey depth to M-dwarfs,
LSST should revolutionize the use of this technique to probe
Interstellar dust.

% --------------------------------------------------------------------

% \subsection{Target measurements and discoveries}
\subsubsection{Target measurements and discoveries}
\label{sec:\secname:targets}

LSST will be in a unique position to measure the changes in the
observed reddening vector due to $R_V$ variations due to its superb
photometric accuracy.  Both of the dust survey techniques mentioned
above can be used on LSST data, and perhaps other methods will be
developed before the start of survey operations.

When focusing on dust in the ISM (as
opposed to time-domain studies, e.g., dust around star-forming
regions or young stars), the main drivers of feasibility are
coverage of the few degrees around the Plane with sufficient photometric depth
and accuracy. This project is less affected by crowding than other
projects requiring photometry in the Plane owing to its use of blue
filters and the high extinction.  Nonetheless, quantiative estimates of
the expected photometric accuracy in coadded $u$ and $g$ images at low
Galactic latitude are desirable.

Production of a 3-D map of the dust component of the ISM based on LSST
photometry will tell us how much dust is present, what type it is, and
where it is along the line of sight.  The latter concern brings in
issues of how to determine stellar photometric parallaxes ($\mu = m-M$)
under an unknown reddening. The dust maps that are created will consist
of the median and variance of $E(B-V)$ and $R_V$ expressed as functions
of $\mu$ under a suitable binning scheme. We can create simple Figure of
Merit maps that lose the $\mu$ dependency by computing the mean and
variance of the measured variances in $E(B-V)$ and $R_V$ over the $\mu$
bins.

With the possible exception of sightlines towards star formation
regions, studies of interstellar dust are insensitive to the
distribution in time of the visits. In the case of active star formation
regions it is possible that changes in the ISM could be apparent over
the lifetime of the survey. Pushing to fainter magnitudes (which means
observing these fields with better seeing and longer exposures) will be
important, because more stars are required for better statistical
constraints on the model, because more stars are required that lie
behind the dust. In general, the use of broad band photometry requires
attention to the intrinsic SEDs of the background stars in order to
correct for heterochromatic variations in the effective reddening law.

% % --------------------------------------------------------------------
%
% \subsection{Metrics}
% \label{sec:\secname:metrics}
%
% {\bf Metric 1: Uncertainty and bias in $E(B-V)$~estimates as a
%   function of location on-sky.} Dependencies:
%
% \begin{itemize}
%   \item Stellar population throughout the survey %(e.g. Knut / Peter developments; TRILEGAL?);
%   \item Dust map throughout the survey region;
%   \item Scale photometric error predictions for each band from program requirements per exposure;
%   \item Produce formal estimate on the error in extinction and reddening as a function of position on-sky within the survey.
% \end{itemize}
%
%
% % --------------------------------------------------------------------
%
% \subsection{OpSim Analysis}
% \label{sec:\secname:analysis}
%
% Table \ref{tab_SummaryMWDust} summarizes the science Figures of Merit
% for the ISM science cases with LSST.
% % At the present date (April
% % 2016) placeholder rows are given for the FoM's.
%
% \begin{table}
%   \begin{tabular}{l|p{6cm}|c|c|c|c|p{5cm}}
%     FoM & Brief description & {\rotatebox{90}{\opsimdbref{db:baseCadence}}} & {\rotatebox{90}{\opsimdbref{db:opstwoPS}}} & {\rotatebox{90}{\scriptsize{\tt astro\_lsst\_01\_1004} }} &  {\rotatebox{90}{future run 2}} & Notes \\
%     \hline
%     1.1 & \footnotesize{Median (over sight-lines) of the uncertainty in $E(B-V)$} & - & - & - & - & \footnotesize{(Most useful FoM probably a spatial map of the uncertainty.)} \\
%     1.2 & \footnotesize{Variance (over sight-lines) of the uncertainty in $E(B-V)$} & - & - & - & - & - \\
%   \end{tabular}
% \caption{Summary of figures-of-merit for ISM science cases. The best
% value of each FoM is indicated in bold. Runs \opsimdbref{db:baseCadence}
% and \opsimdbref{db:opstwoPS} refer to the Baseline and PanSTARRS-like
% strategies, respectively. Column {\tt astro\_lsst\_01\_1004} refers to a
% recently-completed OpSim run that includes the Plane in Wide-Fast-Deep
% observations. See Section \ref{sec:MW_Disk}. }
% \label{tab_SummaryMWDust}
% \end{table}
%
%
% --------------------------------------------------------------------

%\subsection{Discussion}
%\label{sec:\secname:discussion}

%Discussion: what risks have been identified? What suggestions could be
%made to improve this science project's figure of merit, and mitigate
%the identified risks?

% ====================================================================

\navigationbar
