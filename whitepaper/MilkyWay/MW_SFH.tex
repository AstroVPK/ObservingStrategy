% ====================================================================
%+
% SECTION:
%    section-name.tex  % eg lenstimedelays.tex
%
% CHAPTER:
%    chapter.tex  % eg cosmology.tex
%
% ELEVATOR PITCH:
%    Explain in a few sentences what the relevant discovery or
%    measurement is going to be discussed, and what will be important
%    about it. This is for the browsing reader to get a quick feel
%    for what this section is about.
%
% COMMENTS:
%
%
% BUGS:
%
%
% AUTHORS:
%    Phil Marshall (@drphilmarshall)  - put your name and GitHub username here!
%-
% ====================================================================

\section{Star Formation History of the Milky Way}
\def\secname{MW_SFH}\label{sec:\secname} % For example, replace "keyword" with "lenstimedelays"

\noindent{\it Peregrine M. McGehee} % (Writing team)

% This individual section will need to describe the particular
% discoveries and measurements that are being targeted in this section's
% science case. It will be helpful to think of a ``science case" as a
% ``science project" that the authors {\it actually plan to do}. Then,
% the sections can follow the tried and tested format of an observing
% proposal: a brief description of the investigation, with references,
% followed by a technical feasibility piece. This latter part will need
% to be quantified using the MAF framework, via a set of metrics that
% need to be computed for any given observing strategy to quantify its
% impact on the described science case. Ideally, these metrics would be
% combined in a well-motivated figure of merit. The section can conclude
% with a discussion of any risks that have been identified, and how
% these could be mitigated.

LSST gives the opportunity to survey extensive areas
around star formation regions in the Southern hemisphere. Among
others, it would allow to study the Initial Mass Function down to the
sub-stellar limit across different environments. Young stars are
efficiently identified by their variability.

LSST will increase the sample size for detailed follow-up observations due its ability to survey
star formations at large heliocentric distances and to detect variability in embedded and highly
extincted young objects that would otherwise be missed in shallower surveys. During its operations
LSST will also provide statistics on the durations of high states, at least for the shorter duration
EXor variables.

% --------------------------------------------------------------------

\subsection{Target measurements and discoveries}
\label{sec:\secname:targets}

Variability is one of the distinguishing features of pre-main sequence stars and can result from a
diverse collection of physical phenomena including rotational modulation of large starspots due to
kiloGauss magnetic fields, hot spots formed by the impact of accretion streams onto the stellar
photosphere, variations in the mass accretion rate, thermal emission from the circumstellar disk,
and changes in the line of sight extinction. These physical processes generate irregular variability
across the entire LSST wavelength range (320–1040 nm) with amplitudes of tenths to several
magnitudes on timescales ranging from minutes to years and will be detectable by LSST.

The natural of the variability in young stars changes with evolutionary status. For the youngest
stars still undergoing significant mass accretion, FU Orionis and related outbursts can occur 
due to circumstellar disk instabilities. As the natal environment dissipates and the accretion
rates drops, the stars take on a Classical T Tauri appearance where the variability is 
primarily due to changes in the accretion flow and rotational modulation of hot spots resulting
from accretion shocks on the protostellar photosphere. Also present are the signature of
cool spots arising from strong magnetic fields. This cool spot rotational modulation is 
responsible for the variability in the disk-less, and older, Weak-line T Tauri stars.

{\bf CTTS and WTTS material goes here.}

Due to its sensitivity and anticipated ten-year operations lifetime, LSST will also address the issue
of the eruptive variability found in a rare class of young stellar objects - the FUor and EXor stars.
FUor and EXor variables are named after the prototypes FU Orionis (Hartmann \& Kenyon 1996)
and EX Lupi (Herbig et al. 2001) respectively. These stars exhibit outburst behavior characterized
by an up to 6 magnitude increase in optical brightness, with high states persisting from several years
to many decades. Both classes of objects are interpreted as pre-main sequence stars undergoing
significantly increased mass accretion rate possibly due to instabilities in the circumstellar accretion
disk. The mass accretion rates during eruption have been observed to increase by 3 to 4 orders
of magnitude over the $\sim 10^{-9}$
to $10^{-7} M_{\odot}$ per year typical of Classical T Tauri stars. Whether
FUor/EXor eruptions are indeed the signature of an evolutionary phase in all young stars and
whether these outbursts share common mechanisms and differ only in scale is still an open issue.

To date only about 10 FUors, whose eruptions last for decades, having been observed to transition
into outburst (Aspin et al. 2009) with the last major outburst being that of V1057 Cyg (Herbig
1977). Repeat outbursts of several EXors have been studied, including those of EX Lupi (Herbig
et al. 2001) and V1647 Ori (Aspin et al. 2009), the latter erupting in 1966, 2003, and 2008. The
outbursts of EXors only persist for several months to roughly a year in contrast those of FUors,
which may last for decades: for example, the prototype FU Ori has been in a high state for over
70 years. These eruptions can occur very early in the evolution of a protostar as shown by the
detection of EXor outbursts from a deeply embedded Class I protostar in the Serpens star formation
region (Hodapp et al. 1996). The observed rarity of the FUor/EXor phenomenon may be due to
the combination of both the relatively brief (less than 1 Myr) duration of the pre-T Tauri stage
and the high line of sight extinction to these embedded objects hampering observation at optical
and near-IR wavelengths.

V1647 Ori is a well-studied EXor found in the Orion star formation region ($m−M = 8$) and
thus is a suitable case study for discussion of LSST observations. 
The
inferred extinction is $A_r \sim 11$ magnitudes which coupled with the observed $r$ range of 23 to nearly
17 during outburst (McGehee et al. 2004) suggest that $M_r$ varies from 4 to −2 magnitudes.
The LSST single visit 5$\sigma$ depth for point sources is $r$ $\sim$ 24.7, thus analogs of V1647 Ori will
be detectable in the $r$ band during quiescence to $(m−M) + A_r = 20.5$ and at maximum light
to $(m − M) + A_r = 26.5$. The corresponding distance limits are 800 pc to 12 kpc assuming
$A_r$ = 11. For objects at the distance of Orion the extinction limits for LSST $r$-band detections of
a V1647 Ori analog are $A_r$ = 12.5 and $A_r$ = 17.5. These are conservative limits as V1647 Ori was
several magnitudes brighter at longer wavelengths ($iz$ bands) during both outburst and quiescence
indicating that the LSST observations in $izy$ will be even more sensitive to embedded FUor/EXor
stars.

Here are the references cited above:\\
Hartmann \& Kenyon 1996, ARA\&A, 34, 207 \\
Herbig et al. 2001, PASP, 113, 1547 \\
Herbig 1977, ApJ, 217, 693 \\
Aspin et al. 2009, ApJ, 692L, 67 \\
Hodapp et al. 1996, ApJ, 468, 861 \\
McGehee et al. 2004, ApJ, 616, 1058 \\


%Describe the discoveries and measurements you want to make.

%Now, describe their response to the observing strategy. Qualitatively,
%how will the science project be affected by the observing schedule and
%conditions? In broad terms, how would we expect the observing strategy
%to be optimized for this science?


% --------------------------------------------------------------------

\subsection{Metrics}
\label{sec:\secname:metrics}

In order to assess the ability of LSST to 1) identify and 2) classify
Young Stellar Objects we need to quantify the variability timescales and amplitudes of
both Class I/II (stars with disks, including Classical T Tauris) and Class III (Weak-line T Tauris). 
Inclusion of eruptive variables (FUor/EXor) is appropriate as well.

In brief, WTTS are quasi-periodic with amplitudes of 0.1 to 0.3 mag
and periods 1 to $\sim$15 days, so their variability is comparable to 
that of $\gamma$ Dor stars. Given the temporal evolution of cool spots, a
period recovery analysis such as shown for RR Lyrae stars is likely difficult. 
The embedded systems and CTTS are irregular variables but have been shown 
to have distinctive colors due to
extinction and the ultraviolet and blue excess arising from accretion shocks.

% --------------------------------------------------------------------

\subsection{OpSim Analysis}
\label{sec:\secname:analysis}

OpSim analysis: how good would the default observing strategy be, at
the time of writing for this science project?

{\bf This is pending on MAF work.}


% --------------------------------------------------------------------

\subsection{Discussion}
\label{sec:\secname:discussion}

Galactic star formation regions are largely found at low Galactic latitudes or within 
the Gould Belt structure. As such study of young stars with LSST is closely tied to other
science goals concerning the Milky Way Disk and is subject to the concerns of both
crowded field photometry and the observing cadence along the Milky Way.

The embedded and Classical T Tauri stars also undergo significant and rapid color changes
due to accretion processes. The ability of LSST to track these variations in color could
be limited by the interval between filter changes.

% ====================================================================

\navigationbar
