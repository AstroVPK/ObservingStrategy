% ====================================================================
%+
% SECTION:
%    sn.tex  
%
% CHAPTER:
%    transients.tex  
%
% ELEVATOR PITCH:
%    Explain in a few sentences what the relevant discovery or
%    measurement is going to be discussed, and what will be important
%    about it. This is for the browsing reader to get a quick feel
%    for what this section is about.
%
% COMMENTS:
%
%
% BUGS:
%
%
% AUTHORS:
%   Federica Bianco (@fedhere)
%
% ====================================================================

\section{Cataclismic Variables}
\def\secname{CVtransients}\label{sec:\secname} % For example, replace "keyword" with "lenstimedelays"

\credit{paulaszkody}, \credit{fedhere} % (Writing team)

Cataclysmic Variables (CVs) encompass a broad group of objects
including novae, dwarf novae, novalikes, and AM CVn systems, all with different
amplitudes and rate of variability. The one thing they all have in
common is active mass transfer from a late type companion to a
white dwarf. The variability ranges from minutes (due to the flickering in
dwarf novae and novalikes, the pulsations in accreting white dwarfs in
the instability strip, or the orbital periods of AM CVn systems), to
hours (for the orbital periods of novae, dwarf novae and novalikes) to
days (for the normal outburst lengths of dwarf novae) to 
weeks (for the outburst length of superoutbursts in short orbital period
dwarf novae and the outburst recurrence time of normal outbursts in short
orbital period dwarf novae) to months (for the outburst recurrence time of 
longer period dwarf novae, various state changes in novalikes and the decline 
in novae) and, finally, to years (for the outburst recurrence timescales of the 
shortest period dwarf novae and the recurrence times in recurrent novae). The 
amplitudes range from tenths of mags for flickering and pulsations to 4 mags 
for normal dwarf novae and changes in novalike states up to 9-15 mags for the 
largest amplitude dwarf novae and regular novae.

These large differences make correct classification with LSST difficult
but necessary in order to reach goals of assessing the correct number
of types of objects for population studies of the end points of
binary evolution. Multiple filters (especially the blue $u$ and $g$) 
along with amplitude and recurrence of variation provide the best
discrimination, as all CVs are bluer during outburst and high states of
accretion. Long term evenly sampled observations can provide indications
of the low amplitude random variability and catch some of the more frequent
outbursts but higher sampling is needed to determine whether an object
has a normal or superoutburst, to catch a rise to outburst or to a
different accretion state or to follow a nova. Novae typically
have rise times of a few days while the decline time and shape provide
information as to the mass, distance and composition. The time to decline
by 2-3 magnitudes is correlated with composition,

***FED: what is a range of time scales for this decline? days? months?

WD mass and location in
the galaxy, thus enabling a study of Galactic chemical evolution.  As with SN, 
the diagnostic power for all these systems rests on color and sampling.   

The metrics rely on a given cadence to provide shape and recurrence time
of large variations that will distinguish between new novae, dwarf novae
outbursts and identify hig vs low states, as well as available blue colors to 
distinguish low amplitude variability that would indicate new pulsators or 
novalikes. The population studies rely on the numbers of long orbital period 
(low amplitude, wide outbursts) vs short orbital period (patterns of short 
outbursts followed by larger, longer superoutbursts) dwarf novae at different 
places in the galaxy, as well as the numbers of recurrent (1-10 yrs) vs normal 
novae (10,000 yrs, about 35/galaxy/yr). Objects particulary worthy of 
discrimination for later followup are the numbers of CVs containing highly 
magnetic white dwarfs. These can be identified by a metric of 10 yrs of data on
a large sample where the magnitude for a majority of the years is a faint (low)
state and a small percentage of time is a bright (high) state, combined with a 
red color (due to cyclotron emission from the magnetic accretion column).
