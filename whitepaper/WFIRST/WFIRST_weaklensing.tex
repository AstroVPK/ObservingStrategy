% ====================================================================
%+
% SECTION:
%    WFIRST_weaklensing.tex
%
% CHAPTER:
%    wfirst.tex
%
% ELEVATOR PITCH:
%    Explain in a few sentences what the relevant discovery or
%    measurement is going to be discussed, and what will be important
%    about it. This is for the browsing reader to get a quick feel
%    for what this section is about.
%
% COMMENTS:
%
%
% BUGS:
%
%
% AUTHORS:
%    Phil Marshall (@drphilmarshall)  - replace with your name and GitHub username!
%-  Jason Rhodes @jasondrhodes
% ====================================================================

\section{Cosmological Weak Lensing with WFIRST and LSST}
\def\secname{\chpname:weaklensing}\label{sec:\secname}

%note to Phil-  I am not sure if there will be another section that serves as an intro to WFIRST, but I am doing that here
%also, I think we should make this about more than just WL.  Can we change the name and focus of the section to the High Latitude Survey.  The driver %is still probably weak lensing!



\credit{jasondrhodes},
{\it and others to follow}

\textbf{Intro to WFIRST}

The Wide Field Infrared Survey Telescope (WFIRST) is a NASA mission that
entered Phase A in February 2016.  WFIRST was the highest recommendation
for large space missions in the 2010 New Worlds New Horizons Decadal
Survey.  That recommendation envisioned a wide-field observatory with
near infrared (NIR) capabilities to complement LSST's optical
capabilities; the Decadal Survey recognized the obvious synergy between
WFIRST and LSST.  WFIRST's design has evolved since 2010 and the design
being pursued for a mid-2020s launch uses an existing $2.4$m telescope
donated to NASA, giving WFIRST capabilities not envisioned by the
Decadal Survey.  WFIRST has 3 primary science objectives:

\begin{itemize}
\item Determine the nature of the dark energy that is driving the
current accelerating expansion of the universe using a combination of
weak lensing, galaxy clustering (including Baryon Acoustic Oscillations
and Redshift Space Distortions), and supernovae type Ia (SN).
\item Study exoplanets through a statistical microlensing survey and via
direct imaging and spectroscopy with a coronagraph.
\item Perform NIR surveys of the galactic and extragalactic sky via a
Guest Observer program.
\end{itemize}

WFIRST will be at L2 to enable the thermal stability required for the
precise astrometric, photometric, and morphological measurements
required for these science goals. The baseline WFIRST mission
architecture is described in detail in the final report of the WFIRST
Science Definition Team (arxiv/1503.03757). The Wide Field
Instrument(WFI) has a NIR focal plane with a $\sim0.28$ square degree
field of view made up 18 4k$\times$4k Teledyne H4RG NIR detectors will
have imaging capabilities from $0.7-2$ microns and grism spectroscopy
capabilities from $1.35-1.89$ microns with $R\sim461\lambda$.  The WFI
also contains an Integral Field Unit (IFU) spectrometer with $R\sim100$
resoluton over the range $0.6-2$ microns for SN follow up. The exoplanet
coronagraph will have imaging ($0.43-0.97$ microns) and spectroscopic
($0.6-0.97 $ microns) capabilities with a contrast ratio of 1 part in a
billion.

WFIRST's  6 year primary mission will have 2 years dedicated to a
$\sim2200$ square degree High Latitude Survey (HLS) for weak lensing and
galaxy clustering,  1 year of microlensing observations divided into 6
seasons, $0.6$ years of SN search and follow-up, one year dedicated to
the coronagraph and 1.4 years dedicated to competitively selected Guest
Observer observations. WFIRST has no expendables that would prevent an
extended mission of 10 years or longer, and an extended mission would be
given over entirely to Guest Observer observations.

\textbf{WFIRST's High Latitude Survey (HLS)}

WFIRST's HLS will cover 2200 square degrees in 4 NIR photometric filters
(3 of which will be sufficiently sampled for weak lensing shape
measurements) and NIR grism spectroscopy.  The benefits of overlapping
spectroscopic and photometric surveys for dark energy constraints and
systematics mitigation are strong.  The primary scientific driver of the
photometric portion of the WFIRST HLS is weakg gravitational lensing,
but there is a wide range of ancillary science that will be possible
with the publicly available WFIRST HLS data (see for instance, the SDT
report mentioned above).  However, the requirements on the HLS are
largely set by constraints from weak lensing measurements.  Each galaxy
in the WFIRST weak lesing survey needs to have an accurate photometric
redshift.  This requires optical photometry that reaches the depth of
the NIR photometry WFIRST will acquire ($J~27AB$).  \emph{Thus, the
WFIRST weak lensing survey will require the full  10-year LSST depth in
4 optical bands for optimal photometric redsfhift determination}.

There is strong benefit not jsut to WFIRT, but to LSST, in coordinating
observations of the WFIRST HLS survey field. The combination of
full-depth LSST data and WFIRST HLS NIR data will provide the gold
standard in photo-zs.  Furthermore, WFIRST grism observations over the
same area will provide many millions of high quality slitless spectra
and WFIRST’s IFU can be run in parallel with WFI observations to provide
many more very accurate spectroscopic redshifts in the survey area.
Thus, the WFIRST photometric data will help to provide better LSST
photo-zs and  WFIRST will also provide many of the spectra needed for a
training set to calibrate the photo-zs for both missions.  A further
benefit to LSST might be the reduced need for LSST observations at the
reddest end of the LSST wavelength range (the z and y filters), where
both the atmosphere and the physics of CCDs make ground-based
observations less efficient than what WFIRST can achieve. Finally, the
joint processing of LSST and WFIRST data will provide better object
deblending parameters than LSST can achieve alone; WFIRST will be able
to provide a morphological prior for the deblending of LSST images.

% --------------------------------------------------------------------

\subsection{Target measurements and discoveries}
\label{sec:\secname:targets}

We propose an acceleration of the LSST survey over about $10\%$ of the
LSST survey area (the $\sim2200$ WFIRST HLS) such that the full LSST ten
years survey depth is reached on a timescale that maximizes the joint
usefulness of LSST and WFIRST data on that area.  Assuming the two year
WFIRST HLS is taken in the first four years of a WFIRST mission that
launches in 2024, this would require reaching full LSST depth over that
area in $\sim2028$ rather than $\sim2032$. Since the HLS area is roughly
$1/8$ as large as the LSST ``Main Survey"'' region, this could be
achieved by devoting 1.25 years of LSST observations to the HLS area,
assuming that it covers a wide enough range of Right Ascension.  More
practically, it could be achieved by devoting 25\% of LSST observing
time to this area during each of the first 5 years of the LSST survey,
which doubles the time it would naturally be observed during those years
at a modest reduction in coverage of the rest of the Main Survey area
during that time period.   Given existing plans to speed up the LSST
cadence over small sub-areas of the LSST survey, this may only require
coordination of the locations of the accelerated LSST area and the
WFIRST HLS. As LSST and WFIRST progress, there is a mutual benefit in
continuing discussions about the optimal joint observation schedule.

It is possible that the WFIRST data might allow for shallower LSST data
in the reddest LSST filter in the overlap region, and this must be
quantified.


% --------------------------------------------------------------------

\subsection{Metrics}
\label{sec:\secname:metrics}

A simple, first order metric would be the amount of LSST/WFIRST
overlapping survey area that reaches the full LSST depth when the WFIRST
HLS is completed.  Such a metric is straightforward, but not
quantitative until the 2020s, when the WFIRST launch date and survey
plan is more definite.  A slightly more complicated metric could include
the pace at which the overlapping LSST/WFIRST survey areas are both
taken to full depth, since this would make each data set maximally
useful to the US community (or anyone with immediate access to both
WFIRST and LSST data).  WFIRST data is unlikely to have any proprietary
period.  Current plans call for the WFIRST HLS to be conducted in
multple passes, but the exact survey pattern is still undecided, so this
metric is also not quantifiable yet.

There may be some reduced need for the the LSST reddest bands in the
WFIRST HLS overlap area, which should also be folded into the metric.

% --------------------------------------------------------------------

\subsection{OpSim Analysis}
\label{sec:\secname:analysis}

The default survey strategy would only achieve the full LSST photometric
depth over the WFIRST HLS after 10 years of survey ($\sim2032$).


% --------------------------------------------------------------------

\subsection{Discussion}
\label{sec:\secname:discussion}

Increasing the cadence of the LSST survey over $\sim10\%$ of the LSST
survey has science benefits that go far beyond the LSST/WFIRST synergy
described here.  There are benefits to certain aspects of time-domain
science.  Every effort should be made to coordinate all discussions of
increased survey cadence (resulting in full LSST depth well before 10
years) over sub-areas of the LSST survey footprint.  Specific attention
should be paid to whether the accelerated portions of the LSSt survey
can completely overlap the WFIRST HLS, and whether the position of the
WFIRST HLS can be determined, in part, by other science drivers within
LSST.  This will require close LSST and WFIRST coordination at the
Project levels.


% ====================================================================

\navigationbar
