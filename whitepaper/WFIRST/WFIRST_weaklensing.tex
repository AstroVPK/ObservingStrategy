% ====================================================================
%+
% SECTION:
%    WFIRST_weaklensing.tex
%
% CHAPTER:
%    wfirst.tex
%
% ELEVATOR PITCH:
%
%
% AUTHORS:
%    Jason Rhodes @jasondrhodes
%-
% ====================================================================

\section{Cosmology with the WFIRST HLS and LSST}
\def\secname{\chpname:weaklensing}\label{sec:\secname}


\credit{jasondrhodes}

Overview Questions  %added by jasondrhodes 8-14-2016


\begin{enumerate}
\item \textbf{Can you place constraints on the tradeoff between the sky coverage and coadded depth?}
WFIRST requires the full 10 year LSST depth over the $\sim$ 2000 square degree WFIRST High Latitude Survey.
\item \textbf{Can you place constraints on the trade between uniformity of sampling and frequency of
sampling?} The WFIRST HLS synergy does not place constraints on the uniformity of time sampling.
\item \textbf{Can you place constraints on the tradeoff between the single-visit depth and the number
of visits?} The WFIRST HLS only places requirements on the total depth in the 5 LSST photometric filters.
\item \textbf{Can you place constraints on the Galactic plane coverage (spatial coverage, temporal
sampling, visits per band)?} The WFIRST HLS will not be in the Galactic plane.
\item \textbf{Can you place constraints on the fractions of observing time allocated to each band?}
There are multiple solutions to the allocation of depth to each of the 5 LSST and 3 WFIRST bands that will be used for optimzed photometric redshifts. This is a area of ongoing study.  
\item\textbf{ Would your science benefit from a special cadence prescription during commissioning or
early in the survey, such as: acquiring a full 10-yr count of visits for a small area (either in all or in selected bands); a greatly enhanced cadence for a small area?}
The WFIRST HLS synergy would be globally maximized (for both LSST and WFIRST) if the full LSST depth in all 5 bands is reached in the first 5 years of LSST operations.
\item \textbf{Do you have constraints for sampling of observing conditions (e.g. seeing, dark sky,
airmass), possibly as a function of band, etc.?}
The benefit to LSST of having high precision space based galaxy shape measurements would be maximized if the observing conditions allowed for the best possible LSST shapes for cross-calibration of shear measurements.
\item \textbf{Do you have science drivers that would require real-time exposure time optimization
to obtain nearly constant single-visit limiting depth?}
No.
\end{enumerate}

WFIRST's High Latitude Survey (HLS) will cover
2200 square degrees in 4 NIR photometric filters
(3 of which will be sufficiently sampled for weak lensing shape
measurements) and NIR grism spectroscopy.  The benefits of overlapping
spectroscopic and photometric surveys for dark energy constraints and
systematics mitigation are strong.  The primary scientific driver of the
photometric portion of the WFIRST HLS is weak gravitational lensing,
but there is a wide range of ancillary science that will be possible
with the publicly available WFIRST HLS data (see for instance, the SDT
report mentioned above).  However, the requirements on the HLS are
largely set by constraints from weak lensing measurements.  Each galaxy
in the WFIRST weak lensing survey needs to have an accurate photometric
redshift.  This requires optical photometry that reaches the depth of
the NIR photometry WFIRST will acquire ($J~27AB$).  \emph{Thus, the
WFIRST weak lensing survey will require the full  10-year LSST depth in
4 optical bands for optimal photometric redsfhift determination}.

There is strong benefit not just to WFIRST, but to LSST, in coordinating
observations of the WFIRST HLS survey field. The combination of
full-depth LSST data and WFIRST HLS NIR data will provide the gold
standard in photo-zs.  Furthermore, WFIRST grism observations over the
same area will provide many millions of high quality slitless spectra
and WFIRST's IFC can be run in parallel with WFI observations to provide
many more very accurate spectroscopic redshifts in the survey area.
Thus, the WFIRST photometric data will help to provide better LSST
photo-zs and  WFIRST will also provide many of the spectra needed as a
training set to calibrate the photo-zs for both missions.  A further
benefit to LSST might be the reduced need for LSST observations at the
reddest end of the LSST wavelength range (the $z$ and $y$ filters), where
both the atmosphere and the physics of CCDs make ground-based
observations less efficient than what WFIRST can achieve. Further work is needed to quantify this benefit, especially as the WFIRST proposed filter set is evolving.Finally, the
joint processing of LSST and WFIRST data will provide better object
deblending parameters than LSST can achieve alone; WFIRST will be able
to provide a morphological prior for the deblending of LSST images.

% --------------------------------------------------------------------

\subsection{Measuring Dark Energy Parameters}
\label{sec:\secname:targets}

The goal in this science project would be to measure Dark Energy parameters from various weak lensing
probes, capitalizing on the improved photometric redshifts that a joint
analysis of LSST and WFIRST photometry would provide. A useful
Figure of Merit is the usual Dark Energy Task Force figure of merit,
quantifying the available precision on the equation of state
parameters $w_0$ and $w_a$. For example, we are interested in
improvements in the weak lensing DE FoM as the LSST photometric redshifts
and galaxy shapes are improved over the whole LSST survey area
via joint analysis with WFIRST. We are also interested in increasing the
WFIRST DE FoM as quickly as possible.
Indeed, the basic problem we face is one of timing: being able to combine
the WFIRST data with the LSST sooner will accelerate the production of
cosmological results.

Therefore, we propose an acceleration of the LSST survey over about $10\%$ of the
LSST survey area (the $\sim2200$ WFIRST HLS) such that the full LSST ten
years survey depth is reached on a timescale that maximizes the joint
usefulness of LSST and WFIRST data on that area.  Assuming the two year
WFIRST HLS is taken in the first four years of a WFIRST mission that
launches in 2024, this would require reaching full LSST depth over that
area in $\sim2028$ rather than $\sim2032$. Since the HLS area is roughly
$1/8$ as large as the LSST ``Main Survey"'' region, this could be
achieved by devoting 1.25 years of LSST observations to the HLS area,
assuming that it covers a wide enough range of Right Ascension.  More
practically, it could be achieved by devoting 25\% of LSST observing
time to this area during each of the first 5 years of the LSST survey,
which doubles the time it would naturally be observed during those years
at a modest reduction in coverage of the rest of the Main Survey area
during that time period.   Given existing plans to speed up the LSST
cadence over small sub-areas of the LSST survey, this may only require
coordination of the locations of the accelerated LSST area and the
WFIRST HLS. As LSST and WFIRST progress, there is a mutual benefit in
continuing discussions about the optimal joint observation schedule.


A simple, first order diagnostic metric would be the amount of LSST/WFIRST
overlapping survey area that reaches the full LSST depth when the WFIRST
HLS is completed.  Such a metric is straightforward, but not
able to be meanigfully encoded until the 2020s, when the WFIRST launch date and survey
plan is more definite.  Strawman survey plans could, however, be
evaluated to help with LSST schedule planning.
A slightly more complicated metric could include
the pace at which the overlapping LSST/WFIRST survey areas are both
taken to full depth, since this would make each data set maximally
useful to the US community (or anyone with immediate access to both
WFIRST and LSST data).  WFIRST data is unlikely to have any proprietary
period.  Current plans call for the WFIRST HLS to be conducted in
multiple passes, but the exact survey pattern is still undecided, so this
metric is also not quantifiable yet.

There may be some reduced need for the the LSST reddest bands in the
WFIRST HLS overlap area, which should also be folded into the metric.
We note that
the default survey strategy would only achieve the full LSST photometric
depth over the WFIRST HLS after 10 years of survey ($\sim2032$).



% % --------------------------------------------------------------------
%
% \subsection{OpSim Analysis}
% \label{sec:\secname:analysis}
%
%
% % --------------------------------------------------------------------

\subsection{Discussion}
\label{sec:\secname:discussion}

Increasing the cadence of the LSST survey over $\sim10\%$ of the LSST
survey has science benefits that go far beyond the LSST/WFIRST synergy
described here: an observing strategy that met the joint WFIRST/LSST cosmology goals could
also provide the kind of ``rolling cadence'' favored by other science teams.
There are benefits to certain aspects of time-domain
science.  Every effort should be made to coordinate all discussions of
increased survey cadence (resulting in full LSST depth well before 10
years) over sub-areas of the LSST survey footprint.  Specific attention
should be paid to whether the accelerated portions of the LSST survey
can completely overlap the WFIRST HLS, and whether the position of the
WFIRST HLS can be determined, in part, by other science drivers within
LSST.  This will require close LSST and WFIRST coordination at the
Project levels.


% ====================================================================

\navigationbar
