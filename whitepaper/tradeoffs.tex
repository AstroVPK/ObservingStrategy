% ====================================================================
\chapter[Tensions and Trade-offs]{Tensions and Trade-offs}
\def\chpname{tradeoffs}\label{chp:\chpname}

Chapter editors:
\credit{StephenRidgway},
\credit{ivezic}

\section*{Summary}
\addcontentsline{toc}{section}{~~~~~~~~~Summary}

Executive summary goes here, highlighting the primary conclusions from
the chapter's science cases. This should be abstract length, no more:
say, 200 words.

% --------------------------------------------------------------------

\section{Introduction}

The LSST survey will be carried out with physical and operational
constraints that will impact all science objectives.  These include
design limitations, such as the aperture of the telescope, or the
limited number of filter changes that can be supported during the
lifetime of the survey.  They include natural constraints, such as the
quantity of useful observing time in 10 years. They include practical
constraints such as the detector noise and readout time.

The LSST observing schedule can be designed, to some extent, to minimize
the impact that these limitations have on any one or few science
objectives. As the science objectives become more numerous and more
complex, the optimization becomes more difficult and the chances
increase that significant compromises may be required.

In chapters 3-9 of this report, science objectives are described, and
for each, metrics, and in some cases merit functions, are designed to
represent quantitatively the interests of that topic in a schedule
optimization.  Not all of these metric sets are fully worked out, and in
most cases they are provisional pending further analysis and community
review and input.  They do suffice to bring attention to many special
requirements.

The design of the LSST scheduler, and of the algorithms that will select
the visit sequences, has a considerable distance to go before hard
cadence questions must be confronted and resolved.  However, it is
already possible to survey the reach of science needs, and to identify
areas of competition which may become candidates for careful trades and
decisions in the years before the survey begins.

In this chapter, we review the possible tensions that are now evident,
and where tradeoffs may become necessary.

Most potential tensions within the main survey concern temporal sampling
for variable targets.  Tensions among static science objectives, and
between static science and variable science may be less likely and mild.

The strongest point of tension may prove to be between mini-surveys and
the main survey.


% --------------------------------------------------------------------

\section{Variable Targets - Where Is The Tension?}

Strictly periodic targets are relatively neutral to cadence speed, since
successive periods can be combined to improve phase coverage steadily
through the survey.  The only odd cases are ultra-short ($<~ $1 minute)
and ultra long ($> $10 years) periods, and periods very, very close to
one sidereal day.  However, with precision measurements over a long
term, some of the very interesting results for periodic variables will
be in period drifts or slight deviations from periodicity. Furthermore,
even periodic targets benefit from an early interval of higher frequency
sampling, at least in some sky regions, as this can accelerate the ramp
up of the science.

By far the majority of variables and transients, stellar and galactic,
are not periodic. For these, study will be greatly simplified (or may
absolutely require) sufficient sampling within an interval that depends
on the target type. One would like to satisfy the sampling theorem, with
visits at twice the frequency of the highest frequency content, but this
is only a conceptual guide - knowledge of, and experience with, the
targets and the science objectives can provide practical criteria.

A truly uniform cadence provides a revisit rate of one visit pair every
16 days (in r or i filters), or every 3.7 days (in any filter) -
assuming a 5 month observing season.  This is a sparse sample rate for
many variable types.  Achieving higher sample rates requires (possibly
very strong) deviations from complete uniformity.  Thus the obvious -
rapid cadences cannot apply everywhere all the time. Rapid cadences must
be designed, executed selectively, are bounded by the number of visits
available, and must be coordinated with all other such cadences as well
as more general survey requirements.

\subsection{Examples}

Several examples will illustrate the diversity of cadences that are
represented in the science programs described in this white paper.

QSO variability tends to be ``red''.   A uniform distribution of the
LSST visits, with minimal seasonal gaps, provides fairly good support
for identifying QSOs from variability pattern.

For a SN, sufficient sampling must be acquired during the life of the
event. A good cadence in at least one filter is required to support
classification, and multicolors to support photometric redshift
determination.  A uniform LSST  cadence, even with large seasonal gaps,
does not provide a sufficient sampling rate for SN science - an
enhancement of ~2X or greater is strongly requested.

To determine the rotational period of stars with spots, sampling must
resolve light variations sufficiently to constrain periodicity within
the spot lifetime (weeks). This cadence is much more rapid than provided
by a uniformly distributed WFD visit pattern.

Flaring stars and interacting binaries  may show dramatic flux changes
in minutes to hours, and correct identification of such events may
require several data points, and possibly more than one filter, on a
similar time scale.

The solar system small body case is particularly complex.  The science
is one of the main LSST drivers.  Detection of PHAs has a non-scientific
and even political component. Asteroid confusion can interfere with
transient discovery. The density of targets is a strong function of
position on the sky.  Characterization of solar system objects, by
determination of orbits, requires visit patterns on short timescale
($\approx$ hour return) and intermediate time scale ($\approx$2 weeks) -
long timescale confirmation occurs naturally later in the survey.  The
number and pattern of rapid revisits required for positive
identification depends strongly on the false positive rate, which cannot
yet be predicted with confidence.

\subsection{How to provide a range of cadence speeds}

The problem of sampling diverse events was of course recognized very
early in survey planning. Previous cadence development has explored the
following special cadence options:

Rapid revisits - this feature was introduced for study of solar system
bodies and most schedule simulations give high priority to acquiring
visits in pairs with $~$30 minute separation.  Experiments have been
done with triples, in case that should prove necessary for asteroid
characterization.  The possibility of using a different revisit pattern
in different parts of the sky (e.g. less frequent away from the
ecliptic) has been mentioned.  Different patterns for different filters
(e.g. not using pairs for u-filter visits) has been suggested, but not
yet investigated.  The use of visit pairs is clearly a very large impact
decision, for practical purposes reducing by 2X the effective revisit
rate for other targets. However, rapid pairs is very effective for
measuring brightness gradients for rapidly varying objects, and thus
particularly valuable for the difficult problem of characterizing blank
sky transients.

Mini-surveys -  Mini-surveys can include special cadences. The deep
drilling concept utilizes rapid visit sequences to achieve greater depth
without saturation of detector wells, giving sky-limited true time
series with $\approx$30 second sampling steps.  The possibilities for
mini-surveys are limitless, but of course they are bounded by the amount
of time available outside the main survey. The trade between
mini-surveys and the main survey is discussed below.

Rolling cadence - a rolling cadence allows for the possibility of
re-deploying visits within the main survey so as to respond to special
cadence demands without compromising main survey goals (or, perhaps in
principle, trading against main survey goals in a measured and optimum
way). Rolling cadences are discussed in Chapter 2.  As an example, the
average 9 visit pairs per year in the r filter, which would be
distributed over a season in a uniform survey, could be distributed over
2 months, 1 month, 1 week, or 1 day, in a rolling cadence (leaving no
visits in r for the rest of the season).  Or, more conservatively for
the main survey, half (4-5 visit pairs) could be spent in an enhanced
visit rate, reserving the other half to maintain visit pairs in the rest
of the season.  Also, a rolling cadence can concern any number of
filters - for example, one filter could be used to provide short bursts
of rapid sampling, while other filters could maintain a uniform
distribution.  Different rolling cadences can be used in different parts
of the sky, or at different times during the survey.  There is an
immense range of possibilities for rolling cadence, and the surface has
barely been scratched.

Commissioning survey - the highest priority of the commissioning
schedule is - of course - commissioning.  A second objective is to
demonstrate telescope operation in the planned survey mode - presumably
including main survey, deep drilling, rolling cadences, etc.  There have
been ambitious suggestions, going beyond these basics, such as
integrating some fields to the full survey depth, or acquiring some
special cadences.  However, there is no assurance that any of these will
be possible, as they have lower priority than the formal commissioning
objectives.

\subsection{Other options for special cadences}

Pre-survey options - there are a number of survey instruments (CTIO,
CFHT, Subaru) that can easily reach the single visit depth of LSST.
These resources could be used to explore limited sky regions
($\approx$1\% of the LSST sky) with cadences that are planned for LSST
(or cadences that are not planned for LSST), providing touchstone
datasets especially for the more common target types that will dominate
the survey.

Twilight survey - chapter 10 describes a concept for twilight data
acquisition, using short exposures to tolerate bright sky.  This time is
not required for current LSST science, and in principle could be
allocated to z,y filters in short bursts (20 minutes) of fast cadence
($<~$15 seconds) imagery, within the sensitivity limits of the twilight
sky.

Follow-up - LSST is, in large part, a variable discovery tool.  It is
not realistic to expect LSST to provide its own follow-up for all
possible target types and characteristics. Fortunately, many of the most
useful discoveries will be bright enough to follow-up with smaller, more
accessible apertures.  Follow-up can be far more customized to the
science needs than a general purpose LSST cadence.  Faint targets of
sufficient value may likewise merit followup with exceptional  ground
and space-based resources.

Post-survey options - will LSST operate for more than 10 years? It's
possible, maybe even probable, but at present too speculative to plan
around.

\subsection{Frequency of filter changes}

Multi-color visits are a very special case for LSST.  Moving the large
(huge) optical filters involves substantial structure and mechanisms.
While filter change time is not fully characterized, it will be slow
enough that filter change frequency competes directly with efficiency.
Furthermore, the mechanisms have a finite lifetime and are not designed
to allow an indefinite number of changes.  These are practical
limitations of the facility. Combined, they ensure that ``rapid''
multi-color sequences will be the exception rather than the rule.
Science objectives which need near-simultaneous multi-color photometry
will be in competition for a limited resource.

With a limited number of filter changes per night, it is possible for
visit pairs to be acquired (sometimes) in different filters.  Deep
drilling fields which have multiple filters in a single visit will be
close in time (all exposures in one filter will be obtained, then the
filter changed and all the exposures in the next filter obtained).
Other exceptional targets, e.g. certain ToO observations, might be
planned with rapid multi-filter sequences.

For periodic variables, simultaneous multi-color photometry is a
convenience.  For non-periodic or transient targets, it is desirable. It
is important to identify when and if rapid filter changes are essential.

\subsection{Tension between rapid and slow cadences}

In summary, we can readily identify competing demands for very different
cadences, including fast cadences in multiple ranges. For characteristic
times ranging from  $<\approx$1 minute to $~\approx$1 month, a uniform
visit distribution cannot be fully satisfactory, and in some cases it
may be concluded to be totally unsatisfactory.  A number of concepts for
alternate cadences are available.  None can provide rapid cadence all
the time over all the sky. It may be possible to provide cadences
matched to most requirements over part of the sky all of the time, and
over all of the sky at some time. For transient targets, a complex
survey cadence may obtain limited duration but ``appropriate'' sampling
of a fraction of the actual events, with the fraction TBD, but
inevitably significantly less than one.

The tension in scheduling is, first and foremost, not between competing
science objectives, but between science objectives and limited
scheduling flexibility The confrontation between science requirements
and schedule performance leans on the metrics and merit functions that
are the major goal of this white paper.  It should be clear from the 10
foregoing chapters that the difficult goal of metrics analysis is not in
describing sampling for the science, which is ``easy''.  The more
difficult part is in determining the number of science targets for which
adequate sampling can be provided by a simulation, and perhaps the
greatest challenge is determining how many such targets with the
specified sampling are required for a science objective.  It is only
when this step has been accomplished for a large part of the science
that competition between the science objectives can become a
quantitative process.

% --------------------------------------------------------------------

\section{Static Target Science - Is There Any Tension?}

The needs of static target science appear to have fewer points of
potential tension among them than variable targets.  The major
cadence-related concerns are:

Photometry - the best photometric performance will be achieved after the
calibration has been closed around the sky with superior image quality
and superior photometric quality visits to every field.  While this is
probable due to randomization of conditions over 10 years, the sooner
that it is achieved during the survey, the sooner high quality
photometry will be available.  This could be a target of active schedule
control, with corresponding decreasing flexibility in some other
schedule variables.

Astrometry - both proper motions and parallaxes are served by any
schedule that spreads visits well over the duration of the survey.
Parallaxes benefit from observing at a range of hour angles, which is
slightly in competition with the preference to observe at small airmass
for best image quality, but typical simulated schedules show good
astrometric performance. A rolling cadence that moved a significant
fraction of visits from one time period of the survey to another could
impact the astrometric performance (either for better or for worse),
though as long as the fraction of visits concerned was small, the effect
would correspondingly be small.

Homogeneity - an  example of required homogeneity is image quality.
Just as the atmosphere allows a range of image quality during a night
and from night to night, each point on the sky will be observed with a
range of image quality.  To enable understanding of selection effects,
and to compare sky regions on an even playing field, it is desirable
that for each filter, all parts of the sky should be observed with a
similar distribution of image quality, and in particular with similar
best image quality. Achieving homogeneity of conditions actively could
be quite challenging, but simulations show that with a large number of
independent visits, it occurs naturally to good approximation.  Any
cadence that relied on concentrated bursts of visits in a short interval
would tend to reduce the spread of conditions observed. However, such an
extreme has not been proposed or studied

Randomization - closely related to homogeneity, randomization is means
of achieving homogeneity in some observing parameters.  Examples are the
projected angles of camera and telescope optics on the sky. These are
less random than sky conditions, as they depend on instrument setup and
schedule history.  Simulations show that optics angles are well
randomized passively (i.e. without scheduler optimization) for most
points on the sky, but not for all.  Randomization could be improved,
for example by actively running the camera rotator when advancing from
one sky position to the next, in order to populate under represented
camera angles. The rotation takes time, and could reduce the overall
efficiency of the survey.  Only simulations can explore the impact of
these additional mechanical motions

Dithering - dithering of visits is a powerful method of improving
homogeneity of sky coverage passively. Few compromises have been
identified with dithering thus far.  Dithering for small regions has a
price. Imagine the loss in depth due to large dithers with a single FOV,
e.g. a deep drilling field (this has not been proposed).  Due to this
effect, certain rolling cadences can have a potential small loss of
efficiency or efficacity when implemented with dithering.

The foregoing shows that within the static science domain, there are few
and mild points of tension and potential competition.

% --------------------------------------------------------------------

\section{Tension Between Static And Variable Science}

For the most part, the tensions between static science and variable
science are modest and easily understood.  A variable-driven cadence
that requires special timing of visits may result in loss of efficiency
due to increased slew times, or observing under less optimum conditions
(larger airmass).  Special cadences are likely to reduce randomization
and homogenization to a small degree. However, except for very
aggressive cadence implementations, these are second-order effects.
Furthermore, they are measurable with simple metrics - the impact of
variable science schedule considerations on static science should be
small, and it can be readily quantified.

% --------------------------------------------------------------------

\section{Mini-surveys and Main Survey - Tension for Sure}

The LSST proposal and current plan allow a fraction of the total survey
time for mini-survey. These may cover special sky regions or cadences.
The fraction 10\% has been carried for mini-surveys.  This is not
necessarily a sacred number. In Chapter 2, current scheduling experience
has shown that the main survey program, to design depth, can be
accomplished in $\approx$ 85\% of the available time. However,
improvements in simulations could move this estimate up or down.
Adequacy of the design depths could be reconsidered.  And of course the
execution of the survey could encounter unprecedented circumstances.

Proposals for mini-surveys include deep drilling fields, the northern
ecliptic, and the Magellanic clouds. Notional suggestions for deep
drilling fields alone would exceed 10\%.   Most schedule simulations
have allocated a limited number of visits to the galactic plane (based
on the expectation that crowding would limit the useful stacked depth).
However, as detailed in Chapter 4, many areas of galactic science could
benefit from a more aggressive visit plan, perhaps similar to the main
survey.

At present there is no evidence that the trade between main and
mini-surveys will require difficult compromises.  But it is a natural
area of tension, and since is subject to weather experience (not to
mention the evolution of the science) it is likely to be with us through
the life of the survey.

% --------------------------------------------------------------------

\section{Summary and Conclusion}

The likely points of technical and scientific tension in scheduling are
apparent from the schedule simulation experience (Chapter 2) and the
science objectives and metrics (Chapters 3-10).  Static science has
relatively few and mild points of concern.  Variable science has little
and moderate tension with static science.  Variable science has many
points of tension between different variable science objectives, owing
to the wide range of time scales. These lead to contrasting technical
demands.  They may or may not prove to be areas of scientific
competition.

It goes without saying that in essentially all cases, ``more is
better''.  That is however not a sufficient characterization for
schedule planning.

The essential information needed to clarify tensions is the
determination, for each science objective, of three things.
\begin{itemize}
	\item The cadence requirements
	\item For a simulated schedule, the number of instances of targets satisfactorily observed
	\item The number of such targets required for the science
\end{itemize}

This information is the key to calibrating the metrics in terms of
absolute and relative  sufficiency.

% ====================================================================
