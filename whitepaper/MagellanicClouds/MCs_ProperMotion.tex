% ====================================================================
%+
% SECTION:
%    MCs_ProperMotion.tex
%
% CHAPTER:
%    magclouds.tex
%
% ELEVATOR PITCH:
%-
% ====================================================================

% \section{The Proper Motion of the LMC and SMC}
\subsection{The Proper Motion of the LMC and SMC}
\def\secname{\chpname:propermotion}\label{sec:\secname}

\credit{dnidever},
\credit{knutago}

In the last decade work with $HST$ has been able to measure the bulk
tangential (in the plane of the sky) velocities ($\sim$300 km/s) of
the Magellanic Clouds (Kallivayalil et al.\ 2016a,b,2013) and even the
rotation of the LMC disk \citep{2014ApJ...781..121V}. Gaia
will measure precise proper motions of stars to $\sim$20th magnitude
which will include the Magellanic red giant branch stars. LSST will be
complementary to Gaia and measure proper motions of stars in the
$\sim$20--24 mag range that includes Magellanic main-sequence stars
which are far more numerous than giants, and, therefore, more useful
for mapping extended stellar structures. The LSST 10-year survey
proper motion precision will be $\sim$0.3--0.4 mas/yr at LMC
main-sequence turnoff at r$\approx$22.5--23.  This will allow for
accurate measurement of proper motions of {\em individual stars} at the
$\sim$5$\sigma$ level.

Besides measuring kinematics, the LSST proper motions can be used to
produce clean samples of Magellanic stars.
%  clean samples of background
% galaxies (no proper motions) and this is commonly done with $HST$ data
% of
%
In addition, LSST proper motions can be used to improve star/galaxy
separation which is quite significant for faint, blue Magellanic
main-sequence stars.

% streaming motions

% can we do individual LMC stars with LSST, or small groups?

% SRD says want 0.2 mas/yr accuracy over the course of the survey
% 0.2 mas/yr at r=20.5 (similar to Gaia)
% ~0.25 mas/yr at r=22
% ~0.3 mas/yr at r=22.5 LMC turnoff
% ~0.4 mas/yr at r=23
% 1 mas/yr for r=24
% See Figure 21 from Ivezic et al. (2012) or slide 46 of overview-sci-reqs.pdf
%have the astrometric precision to measure the proper motions of individual
%Proper motion cleaning to find the giants?? gaia does that already
%lsst can use proper motion cleaning to do star/galaxy separation as well
%
%The
%The Magellanic Clouds have a large tangential velocity (in the plane of the sky) that has been
%Gaia will be able to see the bright stuff, need lsst to get the MSTO
%gaia/lsst synergy
%
%metric that calculates the proper motion accuracy of LMC MSTO stars at r=23 and calculates the sigma-level of
%the proper motion measurement (2.0 mas/yr / sigma_pm ).

% --------------------------------------------------------------------

% \subsection{Metrics}
\subsubsection{Metrics}
\label{sec:\chpname:metrics}

The natural Figure of Merit for this science case is the precision
with which the proper motion of the Magellanic Clouds can be measured.
This is likely to depend on the following diagnostic metrics:

% metric on surface brightness limit in different parts of the sky
% to MC structure
% could use metric for how much of the Besla stellar debris we can detect
%  even just that region of the sky covered

\begin{itemize}

\item  Single star proper motion precision, possibly quantified as.
the significance level
($\sigma$-level) of the proper motion measurement of one Magellanic MSTO
star (r=23 mag). We would expect this to take values of
$\sim$2.0 mas yr$^{-1}$ / $\sigma_{\rm pm}$.
%  the precision proper motion metric that
%  Metric that calculates the proper motion accuracy of LMC MSTO stars at r=23 and calculates the sigma-level of
%the proper motion measurement (2.0 mas/yr / sigma_pm ).

\item Another useful diagnostic metric would be the surface
brightness limit of the Magellanic structures, using MSTO stars.

\item A metric quantifying how much of the expected Magellanic debris/structure
\citet[from][]{2012MNRAS.421.2109B} model) we can detect would allow the
proper motion science case to be extended to peripheral structures.
This would depend
mostly on the area covered, but we could also use the surface
brightness limit (calculated above) directly.

\end{itemize}

% % --------------------------------------------------------------------
%
% \subsection{Metrics}
% \label{sec:\secname:metrics}
%
% % --------------------------------------------------------------------
%
% \subsection{OpSim Analysis}
% \label{sec:\secname:analysis}
%
% % --------------------------------------------------------------------
%
% \subsection{Discussion}
% \label{sec:\secname:discussion}
%
% % ====================================================================
%
% \navigationbar
