% ====================================================================
%+
% SECTION NAME:
%    photoz.tex
%
% CHAPTER:
%    cosmology.tex
%
% ELEVATOR PITCH:
%    Photometric redshifts are an intermediate data product that comprises 
%    a key input for many investigations of galaxies and cosmology.  They 
%    represent "static science", but we need them to have high quality after 
%    the first year and at each "data release" thereafter.  
%
% COMMENTS:
%
%
% BUGS:
%
%
% AUTHORS:
%   Eric Gawiser (@egawiser)  (just for the outline, hopefully others soon!) 
%-
% ====================================================================
\clearpage
\section{Photometric Redshifts:  Sensitivity to Chromatic Survey Uniformity}
\def\secname{photoz}\label{sec:\secname}

\noindent{\it Andy Connolly, Jeff Newman, Sam Schmidt, TBD???} % (Writing team)



%---------------------------------------------------------------------

\subsection{Observing Strategy Sensitivity:  Color Uncertainties}
\label{sec:\secname:sensitivity}



%---------------------------------------------------------------------

\subsection{Metrics}
\label{sec:\secname:metrics}



%---------------------------------------------------------------------

\subsection{Figure of Merit}
\label{sec:\secname:fom}

\todo{EG} Determine if clusters will be mentioned here or under LSS below or as a separate section

\navigationbar

% ====================================================================
