
% ====================================================================
%+
% NAME:
%    section-name.tex
%
% ELEVATOR PITCH:
%    Explain in a few sentences what the relevant discovery or
%    measurement is going to be discussed, and what will be important
%    about it. This is for the browsing reader to get a quick feel
%    for what this section is about.
%
% COMMENTS:
%
%
% BUGS:
%
%
% AUTHORS:
%    Phil Marshall (@drphilmarshall)  - put your name and GitHub username here!
%-
% ====================================================================

\section{Probing Planet Populations with LSST}
\def\secname{periodicvariables}\label{sec:\secname}

\noindent{\it Mike Lund, Avi Shporer, add your name here} % (Writing team)

% This individual section will need to describe the particular
% discoveries and measurements that are being targeted in this section's
% science case. It will be helpful to think of a ``science case" as a
% ``science project" that the authors {\it actually plan to do}. Then,
% the sections can follow the tried and tested format of an observing
% proposal: a brief description of the investigation, with references,
% followed by a technical feasibility piece. This latter part will need
% to be quantified using the MAF framework, via a set of metrics that
% need to be computed for any given observing strategy to quantify its
% impact on the described science case. Ideally, these metrics would be
% combined in a well-motivated figure of merit. The section can conclude
% with a discussion of any risks that have been identified, and how
% these could be mitigated.

This section describes the unique discovery space for extrasolar planets with LSST.

\subsection{Planets In Relatively Unexplored Environments}
A large number of exoplanets have been discovered over the past few decades, with over 1500 exoplanets now confirmed. These discoveries are primarily the result of two detection methods: The radial velocity (RV) method where the planet's minimum mass is measured, and the transit method where the planet radius is measured and RV follow-up allows the measurement of the planet's mass and hence mean density. Other methods are curently being developed and use to discover an increasing number of planets, inclduing the microlensing method and direct imaging. In addition, the Gaia mission is expected to discover a large number of planets using astrometry (Perryman et al.~2014, http://adsabs.harvard.edu/abs/2014ApJ...797...14P).

The {\it Kepler} mission has an additional almost 4000 planet candidates. While these planet candidates have not been confirmed, the sample is significant enough that planet characteristics can be studied statistically, including radius and period distributions and planet occurrence rates. LSST will extend previous transiting planet searches by observing stellar populations that have generally not been well-studied by previous transiting planet searches, including star clusters, the galactic bulge, red dwarfs, white dwarfs (see below), and the magellanic clouds (see Section~7). Most known exoplanets have been found relatively nearby, as exoplanet systems with measured distances have a median distance of around 80~pc, and 80\% of these systems are within 320~pc (exoplanets.org). LSST is able to recover transiting exoplanets at much larger distances, including in the galactic bulge and the Large Magellanic Cloud, allowing for measurements of planet occurrence rates in these other stellar environments (Lund http://arxiv.org/pdf/1408.2305v2.pdf, Jacklin http://arxiv.org/pdf/1503.00059v2.pdf). Red dwarfs have often been underrepresented in searches that have focused on solar-mass stars, however red dwarfs are plentiful, and better than 1 in 7 are expected to host earth-sized planets in the habitable zone (Dressing http://arxiv.org/pdf/1501.01623v2.pdf).

Another currently unexplored environment where LSST will be able to probe the exoplanet population is planets orbiting white dwarfs (WDs). Such systems teach us about the future evolution of planetary systems with main-sequence primaries, including that of the Solar System. When a WD is eclipsed by a planet (or any other faint low-mass object, including a brown dwarf or a small star) the radius and temperature ratios lead to a very deep eclipse, possibly a complete occultation, where during eclipse the target can drop below the detection threshold. The existence of planets orbiting WDs has been suggested observationally (e.g., Farihi et al.~2009, http://adsabs.harvard.edu/abs/2009ApJ...694..805F; Jura et al.~2009, http://adsabs.harvard.edu/abs/2009AJ....137.3191J; Zuckerman et al.~2010, http://adsabs.harvard.edu/abs/2010ApJ...722..725Z; Debes et al.~2012, http://adsabs.harvard.edu/abs/2012ApJ...747..148D) and theoretically (e.g., Nordhaus et al.~2010, http://adsabs.harvard.edu/abs/2010MNRAS.408..631N). A few brown dwarf companions were already discovered (e.g., Maxted et al.~2006, http://adsabs.harvard.edu/abs/2006Natur.442..543M; Casewell et al.~2012,http://adsabs.harvard.edu/abs/2012ApJ...759L..34C; Littlefair et al.~2006, http://adsabs.harvard.edu/abs/2006Sci...314.1578L, 2014, http://adsabs.harvard.edu/abs/2014MNRAS.445.2106L), and Vanderburg et al.~(2015, http://adsabs.harvard.edu/abs/2015Natur.526..546V; see also Croll et al.~2015, http://adsabs.harvard.edu/abs/2015arXiv151006434C; Gansicke et al.~2016, http://adsabs.harvard.edu/abs/2016ApJ...818L...7G; Rappaport et al.~2016, http://adsabs.harvard.edu/abs/2016arXiv160200740R) recently discovered a disintegrating planetary body orbiting a WD.

While most of the sky that LSST will survey will be at much lower cadences than transiting planet searches employ, a sufficient understanding of the LSST efficiency for detecting planets combined with the large number of targets may still provide significant results. Additionally, the multiband nature of LSST provides an extra benefit, as exoplanet transits are achromatic while many potential astrophysical false positives, such as binary stars, are not. 



% --------------------------------------------------------------------

\subsection{OpSim Analysis}
\label{sec:keyword:analysis}



% --------------------------------------------------------------------

\subsection{Discussion}
\label{sec:keyword:discussion}

% ====================================================================

\navigationbar
