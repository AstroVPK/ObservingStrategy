% ====================================================================
%+
% SECTION:
%    SolarSystem_PHA.tex
%
% CHAPTER:
%    solarsystem.tex
%
% ELEVATOR PITCH:
%    Discovery of PHAs in particular. Discussion of wider 'impacts'.
%
%-
% ====================================================================

\section{Discovery of Potentially Hazardous Asteroid Discovery}
\def\secname{\chpname:phas}\label{sec:\secname}

\credit{ivezic},
\credit{rhiannonlynne}.

The U.S. Congress has given a mandate to NASA to implement a
Near-Earth Object (NEO) Survey program to detect, track, catalog, and
characterize the physical characteristics of near-Earth objects equal
to or greater than 140 meters in diameter\footnote{See
\url{http://www.gpo.gov/fdsys/pkg/PLAW-109publ155/pdf/PLAW-109publ155.pdf}}. The
goal is to achieve a completeness of 90\%. In recent practice, adopted
here, the completeness is evaluated for a subset of NEOs called
Potentially Hazardous Asteroids\footnote{Potentially Hazardous
Asteroids (PHAs) are defined as asteroids with a minimum orbit
intersection distance (MOID) of 0.05 AU or less.} (PHA), with
H$\le$22, where H is the absolute magnitude\footnote{Absolute
magnitude is the magnitude that an asteroid would have at a distance
of 1 AU from the Sun and from the Earth, viewed at zero phase
angle. This is an impossible configuration, of course, but the
definition is motivated by desire to separate asteroid physical
characteristics from the observing configuration.} in the Johnson's V
band.

The discovery criteria for PHAs follows the same guidelines and metrics found in the previous
section, \ref{sec:solarsystem:discovery}, but is worth discussing
separately to focus on its main figure
of merit - completeness for PHAs with H$\le$22 magnitudes.

% --------------------------------------------------------------------

\subsection{Target measurements and discoveries}
\label{sec:\secname:targets}

Using the same range of discovery criteria as in the previous section,
\ref{sec:solarsystem:discovery}, 


%%%%%%%%%%%%%%%%%%%%%%%%%%%
\begin{figure}[t!]
\vskip -1.1in
\includegraphics[angle=0,width=0.56\hsize:,clip]{figs/enigma1189_diffNEOcompleteness.pdf}
\hskip -0.5in
\includegraphics[angle=0,width=0.56\hsize:,clip]{figs/enigma1189_cumNEOcompleteness.pdf}
\vskip -1.2in
\caption{The PHA completeness for \opsimdbref{db:baseCadence}, as a function of the object's absolute
visual magnitude H on the horizontal axes (left: differential completeness at a given H;
right: cumulative completeness for all objects brighter than a given H).
The completeness for H$\le$22 NEOs (those with diameters larger than 140m)  for this
simulation is 73\% (blue line in the right panel).}
\label{fig:baselinePHA}
\end{figure}
%%%%%%%%%%%%%%%%%%%%%%%%%%%

%%%%%%%%%%%%%%%%%%%%%%%%%%%
\begin{figure}[th!]
\vskip -1.2in
\includegraphics[angle=0,width=0.49\hsize:,clip]{figs/diffNEOpairs.pdf}
\includegraphics[angle=0,width=0.49\hsize:,clip]{figs/diffNEOquads.pdf}
\vskip -1.3in
\caption{%
The comparison of differential PHA completeness for the baseline
\opsimdbref{db:baseCadence}, \opsimdbref{
when requiring two detections per night (left) and four detections per night (right).
With two detections per night, all simulations perform similarly but when four
detections per night are required, the simulation that has the largest number
of such sequences (see \autoref{fig:NvisitStats}), performs the best although at an
inferior level compared to the left panel (see also \autoref{fig:baselinePHA}).}
\label{fig:NEOquads}
\end{figure}
%%%%%%%%%%%%%%%%%%%%%%%%%%%


{\bf Analysis Results:}

For baseline reference, the PHA completeness for
\opsimdbref{db:baseCadence} is shown in \autoref{fig:enigmaNEO}. The
baseline cadence achieves a cumulative completeness of 73\% for
H$\le$22 PHAs. This cumulative completeness for H$\le$22 is 17\%
higher than differential completeness at H=22 of 56\% due to
increasing completeness towards smaller H (larger objects). Both
differential and cumulative completeness are relevant metrics: the
former provides more insight in the behavior of a particular
simulation, while the latter is a metric given to NASA by the U.S.
Congress. Analysis of results illustrated in \autoref{fig:NEOquads}
can be summarized as follows:
\begin{itemize}
\item When NEO discovery algorithm requires pairs of visits, all runs
have very similar PHA completeness, with quads run only about 2\%
lower than the baseline (a differential completeness of 56\% at H=22
for \opsimdbref{db:baseCadence})
\item When NEO discovery algorithm requires 4 detections per night,
the simulation with quads achieves a differential completeness of
about 27\% at H=22, or  about 30\% lower completeness than Baseline
Cadence.
\item When NEO discovery algorithm requires 4 detections per night,
Baseline Cadence reaches a differential completeness of about 15\% at
H=22 (some quads are unintentionally produced by chance, see
\autoref{fig:NvisitStats}).
\item When NEO discovery algorithm requires 3 detections per night,
runs which requested triples and quads achieve a differential
completeness of about 40\% at H=22 (corresponding to a cumulative
completeness of about 57\% for H$\le$22).
\end{itemize}

Therefore, going from pairs of visits to triples (both for cadence and
NEO detection) reduces completeness (both differential and cumulative)
for PHAs with H$\le$22 by about 15-20\% (and by about 30\% for quads).

\navigationbar
