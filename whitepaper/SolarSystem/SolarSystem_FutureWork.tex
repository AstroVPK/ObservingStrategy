% ====================================================================
%+
% SECTION:
%    SolarSystem_FutureWork.tex
%
% CHAPTER:
%    solarsystem.tex
%
% ELEVATOR PITCH:
%    Ideas for future metric investigation, with quantitaive analysis
%    still pending.
%-
% ====================================================================

\section{Future Work}
\def\secname{\chpname:future}\label{sec:\secname}

In this section we provide a short compendium of science cases that
are either still being developed, or that are deserving of quantitative
MAF analysis at some point in the future.

% ====================================================================
%
% \section{Deep Drilling Observations}
\subsection{Deep Drilling Observations}
\def\secname{\chpname:dd}\label{sec:\secname}

\credit{rhiannonlynne},
\credit{davidtrilling}

Deep drilling observations provide the opportunity, via digital
shift-and-stack techniques, to discover Solar System Objects fainter
than the individual image limiting magnitude. These fainter objects
will be smaller, more distant, or lower albedo (or some combination of these)
than the general population found with individual images. Discovering smaller
objects is useful for constraining the size distribution to smaller
sizes; this provides constraints for collisional models and insights
into planetesimal formation. More distant objects are interesting in
terms of extending our understanding of each population over a wider
range of space; examples would be discovering very distant
Sedna-like objects or comets at larger distances from the Sun before
the onset of activity. Lower albedo objects may be useful to
understand the distribution of albedos, particularly to look for
trends with size.

Variations on the basic method of shift-and-stack have been used to
detect faint TNOs.
% XXX Allen, Bernstein, Gladman, Fuentes, ? XXX.
Computational limitations on these methods mean that, roughly and in
general for images taken at opposition, images taken over the timespan
of about an hour can be combined and searched for main belt asteroids,
and images taken over the timespan of about 3 days can be combined and
searched for more distant objects like TNOs.

With extragalactic deep drilling fields as in the baseline cadence,
where observations are taken in a series of filters ($g$, $r$, and $i$
would be useful for this purpose) each night, every three or four
days, we could use shift-and-stack to coadd the 50 images obtained in
$gri$ bandpasses in a single night. This would allow detection of
objects about 2 magnitudes fainter than in the regular survey, or
approximately $r=26.5$.

The Solar System Science Collaboration developed a deep drilling
proposal specifically targeted to search for very faint Main Belt
Asteroids (MBAs),  Jupiter Trojans,
and TNOs. This proposal can be summarized as
follows:
\begin{itemize}
\item 9 fields, in a 3x3 contiguous grid block centered on a spot
  where Jupiter Trojans and Neptune Trojans coincide (if possible,
  based on timing)
\item 8 sequences of $\sim1.5$ hour $r$-band exposures per field, in continuous
  observing blocks. Each of these blocks would have a coadded limiting
  magnitude of about $r=27$, letting us push to smaller sizes than
  possible with the general extragalactic deep drilling fields.
\item These 1.5 hour blocks would be spaced apart in time
   \begin{enumerate}
   \item Two blocks acquired on two nights, 1.5 months before the fields come to
     opposition
  \item  Two blocks acquired on two nights when the fields are at
    opposition
  \item Two blocks acquired on two nights, 1.5 months after the fields
    come to opposition
  \item Two blocks acquired on two nights when the fields are at
    opposition again, one year later. 
 \end{enumerate}
\item The location of the fields would be adjusted slightly to account
  for the bulk motion of TNOs in the field, thus letting us follow the
  majority of these very small objects over the course of a year,
  providing fairly accurate orbits. Most of the Jupiter Trojans and
  MBAs would diffuse out of the fields, however we would still have
  approximate sizes from the magnitude and distance estimates provided
  by two nights of observations. 
\end{itemize}

This proposal differs from the general extraglactic deep drilling
fields in that the field selection, observing cadence, and filter
choice is better suited for exploring faint Solar System Objects. More
details are available in the Solar System Collaboration Deep Drilling
whitepaper, \url{https://lsstcorp.org/sites/default/files/WP/Becker-solarsystem-01.pdf}.


% Describe how we will evaluate DD proposal, with TNO population +
% estimate on number of times objects observed (but not doing actual
% shift-and-stack). Need large-i populations to test if useful, probably.

% ====================================================================

\navigationbar
