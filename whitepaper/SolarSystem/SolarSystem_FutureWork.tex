% ====================================================================
%+
% SECTION:
%    SolarSystem_FutureWork.tex
%
% CHAPTER:
%    solarsystem.tex
%
% ELEVATOR PITCH:
%    Ideas for future metric investigation, with quantitaive analysis
%    still pending.
%-
% ====================================================================

\section{Future Work}
\def\secname{\chpname:future}\label{sec:\secname}

In this section we provide a short compendium of science cases that
are either still being developed, or that are deserving of quantitative
MAF analysis at some point in the future.

% ====================================================================
%
% \section{Deep Drilling Observations}
\subsection{Deep Drilling Observations}
\def\secname{\chpname:dd}\label{sec:\secname}

Deep drilling observations provide the opportunity, via digital
shift-and-stack techniques, to discover Solar System Objects fainter
than the individual image limiting magnitude. These fainter objects
will be smaller, more distant, or lower albedo (or some combination of these)
than the general population found with individual images. Discovering smaller
objects is useful for constraining the size distribution to smaller
sizes; this provides constraints for collisional models and insights
into planetesimal formation. More distant objects are interesting in
terms of extending our understanding of each population over a wider
range of space; examples would be discovering very distant
Sedna-like objects or comets at larger distances from the Sun before
the onset of activity. Lower albedo objects may be useful to
understand the distribution of albedos, particularly to look for
trends with size.

Variations on the basic method of shift-and-stack have been used to
detect faint TNOs.
% XXX Allen, Bernstein, Gladman, Fuentes, ? XXX.
Computational limitations on these methods mean that, roughly and in
general for images taken at opposition, images taken over the timespan
of about an hour can be combined and searched for main belt asteroids,
and images taken over the timespan of about 3 days can be combined and
searched for more distant objects like TNOs.

With extragalactic deep drilling fields as in the baseline cadence,
where observations are taken in a series of filters (g, r, and i
would be useful for this purpose) each night, every three or four
days, we could use shift-and-stack to coadd the 50 images obtained in
gri bandpasses in a single night. This would allow detection of
objects about 2 magnitudes fainter than in the regular survey, or
approximately $r=26.5$.
% This is cool and a range of ecliptic
% latitudes is interesting.  But, we would like to do better.
%
% Recap solar system DD white paper.
%
% Describe how we will evaluate DD proposal, with TNO population +
% estimate on number of times objects observed (but not doing actual
% shift-and-stack). Need large-i populations to test if useful, probably.

% ====================================================================

\navigationbar
