% ====================================================================
%+
% SECTION:
%    eruptive.tex
%
% CHAPTER:
%    transients.tex
%
% ELEVATOR PITCH:
%-
% ====================================================================

% \section{LBVs and related non-supernova transients}
\subsection{LBVs and related non-supernova transients}
\def\secname{\chpname:LBVs}\label{sec:\secname}

\credit{nathansmith}

There is a large and diverse class of visible-wavelength transient
sources recognized in nearby galaxies that appear to be distinct from
traditional novae and from SNe, and have often been associated with
the giant eruptions of luminous blue varibles (LBV), such as the 19th
century outburst of $\eta$ Carinae.  Broadly speaking, members of this
class of transients share the common properties that they have peak
luminosities below those of most core-collapse SNe and more luminous
than novae and CVs (absolute magnitudes of roughly $-$9 to $-$15 mag).
They also have H-rich spectra (usually) with relatively narrow lines
that indicate modest bulk outflow velocities of 10$^2$ to 10$^3$ km
s$^{-1}$ (although some have exhibited small amounts of material at
faster speeds).  They tend to evolve on fairly long timescales of
weeks to years (although sometimes they exhibit a quick rise to peak
similar to SNe II-P). This group of transients has gone by many names,
such as LBV eruptions, SN impostors, Type V supernovae,
intermediate-luminosity optical (or red) transients, as well as others
that often include a physical interpretation.  For brevity, these are
often collectively referred to as ``LBVs'', although many of them may
not actually be LBVs.
%This may be largely for historical reasons,
%since LBVs were the first of these to be recognized as a class.  Some
%of the subgroups may be very different from objects like $\eta$
Carinae, however.

Observationally, these eruptions are understood to represent important
and dramatic mass-loss episodes in the lives of massive stars, based
on empirical estimates for the amount of ejected matter.
%Guided
%largely by nearby LBVs with resolved shells, t
These eruptions are
expected to instigate mass loss that is comparable to or more
important than metallicity-dependent winds of massive stars.  This
mode of mass loss, regardless of the mechanism, may be a very
important ingredient in the evolution of massive stars that is
currently not included in stellar evolution models.  Correcting this
is one of the key science drivers in trying to understand the physics
these eruptions.

An important empirical discriminant of subgroups in this class comes
from their progenitor stars.  Some are indeed seen to be very
luminous, blue supergiant stars consistent with traditional LBVs.
Some, however, have somewhat less luminous, heavily dust-obscured
progenitor stars that have been associated with either dust enshrouded
blue or red supergiants, or alternatively, with super-AGB stars of
8-10 M$_{\odot}$, with uncertainty .
%The degeneracy
arises because when the objects are
fully obscured by dust, one cannot actually meaure the star's
temperature, and the bolometric luminosities of super-AGB and red and blue supergiants overlap.  Unfortunately,
cases when we have strong constraints on the quiescent progenitor are
rare, and once they reach their peak luminosity, there is a great deal
of overlap in observed properties.

Theoretically, these eruptions are not understood.  There are many
ideas, but few if any confirmed mechanisms tied to observed
objects.
Some
%previously discussed
theoretical ideas involve (1)
winds driven by super-Eddington instabilities (although the root cause
for suddenly exceeding the Eddington limit remains unexplained), (2)
hydrodynamic explosions caused by deep-seated energy deposition, such
as unsteady nuclear burning, (3) accretion onto companion stars in
binary systems (degenerate or not), (4) mergers in binary and triple
systems, (5) electron-capture SNe, and (6) ``failed SNe'' associated
with a weak explosion and envelope ejection that results from black
hole formation during core collapse.
Because of the relatively low total energy indicated by
radiative luminosities and outflow speeds, these are usually discussed
as non-terminal eruptions, however, the last two are terminal events
that are less luminous and lower energy than normal SNe, and the last
3 should only occur once for a given source.
Together with several well-studied examples that indicate
repeating eruptions, there are indeed many
cases where only one such transient has been seen at the same
position, and some cases where late-time observations suggest that no
source has survived with a luminosity comparable to its progenitor.
%However, there are also several well-studied examples that indicate
%repeating eruptions (multiple repeating transients, multiple nebular
%shells with different ages, etc).
All these theoretical mechanisms
may lead to similar observed phenomena: weak explosions, moderate
luminosities, slow expansion, dusty aftermath, but this class of objects
may represent a mixed-bag of different mechanisms that get lumped
together by default as ``other'' because they are not traditional SNe.

An area of recent interest
is that eruptive non-terminal transients have been observed, in some
cases, to precede much more powerful explosions that are seen as Type
IIn supernovae.  Detectability of SN precursors is discussed in ~\autoref{chp:galaxy}. Even if the pre-SN transients are not observed
directly, pre-SN eruptive mass loss is inferred based on circumstellar
interaction diagnistocs of the SN.  These SN precursors have observed
or inferred properties that are very similar to LBVs and related
transients, 
%.  This may suggest some link between them,
but then again,
most of the LBVs and other SN impostors have been observed for decades
and have not gone SN (yet).  \emph{Being able to distinguish which of these
optical transients are SN precursors and which are not is a major
science driver.}  The amount of mass lost in a precursor eruption may
dramatically alter the type of SN that is observed.  There may also be
a continuum of energies in pre-SN outbursts, extending down to more
normal classes of core-collapse SNe, but these may often go
unrecognized unless the SN is caught very early after explosion.

Rates for these LBV-like eruptions are very poorly constrained,
largely because most previous SN and transient searches with small
telescopes have been optimized for finding more luminous SNe in a
larger volume.  This field begun to change with recent surveys, and will
be revolutionized with LSST.  From discovered examples we have,
numbers are very roughly consistent with a volumetric rate comparable
to that of core-collapse SNe or larger.  %, but with a large error bar.
Rates of individual subclasses are not well constrainted, and limited
information often makes classification into various subgroups
difficult or highly subjective.  The ``rate'' also depends on how
faint the lower limit of inclusions is; evidence suggests that the
brightest events are more rare, and that numbers increase as one moves
to lower luminosity.  At the faint end, it becomes difficult to
distinguish between eruptions and regular variability of LBVs, or
between massive star eruptions and CVs. With deep LSST stacks identifying faint CV in quiescent states this will also change dramatically in the LSST age, with the unvailing of detailed progenitor
information.  Having deep, pre-eruption characterization
of sources at the positions of these eruptive transients (as well as
SN precursors) will likely be a major contribution of LSST.

In terms of timescales, many of the eruptive transients exhibit rise
and decline timescales similar to normal SNe~II-P or II-L, but with
fainter peak luminosity.  For these, observational cadence
requirements will be the same as SNe.  For some eruptive transients,
however, the rise timescales can be very long (rising a few magnitudes
in years).  While LSST's cadence will certainly be fast enough, being
able to discover slowly rising tranients that do not change much from
night to night will be an important metric.

For the faster-rising transients, spectroscopic followup is needed to
discriminate these from normal SNe, and also contextual information
about the host galaxy (and hence, the absolute magnitude) is needed to
differentiate these non-terminal eruptions from Type IIn supernovae
(their spectra look similar, although LBVs do tend to have narrower
lines).  Spectral and color evolution, as well as information about
the progenitor, is needed to distinguish among subgroups within the
class.  Multiwavelength followup is often extremely valuable or even
essential; i.e. mid-IR tells us if an optically invisible source is
cloaked in a dust shell but still quite luminous; Xrays and radio tell
us if an expanding shock wave is the likely source of persistent
luminosity.  For these reasons, nearby cases will continue to be the
most valuable in deciphering the physics of subclasses, whereas the
increased volume in which LSST discovers these fainter transients will
drastically improve our understanding of their rates.  Armed with both
a better understanding of their underlying physics and
characterization, as well as their rates and duty cycles, these
eruptive events can then be incorporated into stellar evolution models
and population synthesis/feedback models.

% % --------------------------------------------------------------------
%
% \subsection{Metrics}
% \label{sec:\secname:metrics}
%
% % --------------------------------------------------------------------
%
% \subsection{OpSim Analysis}
% \label{sec:\secname:analysis}
%
% % --------------------------------------------------------------------
%
% \subsection{Discussion}
% \label{sec:\secname:discussion}
%
% % ====================================================================
%
% \navigationbar
