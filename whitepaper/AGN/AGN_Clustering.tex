% ====================================================================
%+
% SECTION:
%    AGN_Clustering.tex
%
% CHAPTER:
%    agn.tex
%
% ELEVATOR PITCH:
%-
% ====================================================================

% \section{AGN Clustering}
\subsection{AGN Clustering}
\def\secname{\chpname:clustering}\label{sec:\secname}

\credit{ohadshemmer}

Measurements of the spatial clustering of AGNs with respect
to those of quiescent galaxies can provide clues as to how galaxies
form inside their dark-matter halos, and what causes the growth of
their supermassive black holes (SMBHs). The impressive inventory of
LSST AGNs will enable the clustering, and thus the host galaxy halo
mass, to be determined over the widest ranges of cosmic epoch
and accretion power.


% % --------------------------------------------------------------------
%
% \subsection{Target measurements and discoveries}
% \label{sec:\secname:targets}

% Describe the discoveries and measurements you want to make.

% Now, describe their response to the observing strategy. Qualitatively,
% how will the science project be affected by the observing schedule and
% conditions? In broad terms, how would we expect the observing strategy
% to be optimized for this science?

We would consider the 2-point angular correlation function of AGN as our
target measurement.
The LSST cadence will not only affect the overall AGN census and its
$L-z$ distribution, but also the depth in each band as a function of
sky position that can directly affect the clustering signal.

% CROSS REFERENCE TO THE COSMOLOGY CHAPTER'S LSS SECTION! WHAT'S
% DIFFERENT ABOUT AGN CLUSTERING?

% --------------------------------------------------------------------
%
% \subsection{Metrics}
% \label{sec:\secname:metrics}
%
% Quantifying the response via MAF metrics: definition of the metrics,
% and any derived overall figure of merit.
%
%
% % --------------------------------------------------------------------
%
% \subsection{OpSim Analysis}
% \label{sec:\secname:analysis}
%
% OpSim analysis: how good would the default observing strategy be, at
% the time of writing for this science project?
%
%
% % --------------------------------------------------------------------
%
% \subsection{Discussion}
% \label{sec:\secname:discussion}
%
% Discussion: what risks have been identified? What suggestions could be
% made to improve this science project's figure of merit, and mitigate
% the identified risks?
%
%
% ====================================================================

\navigationbar
