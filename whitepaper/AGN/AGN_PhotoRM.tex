% ====================================================================
%+
% SECTION:
%    section-name.tex  % eg lenstimedelays.tex
%
% CHAPTER:
%    chapter.tex  % eg cosmology.tex
%
% ELEVATOR PITCH:
%    Explain in a few sentences what the relevant discovery or
%    measurement is going to be discussed, and what will be important
%    about it. This is for the browsing reader to get a quick feel
%    for what this section is about.
%
% COMMENTS:
%
%
% BUGS:
%
%
% AUTHORS:
%    Phil Marshall (@drphilmarshall)  - put your name and GitHub username here!
%-
% ====================================================================

\section{Photometric Reverberation Mapping}
\def\secname{\chpname:photoRM}\label{sec:\secname}

\credit{ohadshemmer}

% This individual section will need to describe the particular
% discoveries and measurements that are being targeted in this section's
% science case. It will be helpful to think of a ``science case" as a
% ``science project" that the authors {\it actually plan to do}. Then,
% the sections can follow the tried and tested format of an observing
% proposal: a brief description of the investigation, with references,
% followed by a technical feasibility piece. This latter part will need
% to be quantified using the MAF framework, via a set of metrics that
% need to be computed for any given observing strategy to quantify its
% impact on the described science case. Ideally, these metrics would be
% combined in a well-motivated figure of merit. The section can conclude
% with a discussion of any risks that have been identified, and how
% these could be mitigated.

% A short preamble goes here. What's the context for this science
% project? Where does it fit in the big picture?

Photometric reverberation mapping (PRM), measuring the time-delayed
response of either the flux of the broad emission line region (BELR)
lines to the flux of the AGN continuum or between the continuum flux
in one (longer wavelength) band to the continuum flux in another (band
with shorter wavelength), will be one of the cornerstones of AGN
research in the LSST era (e.g., \citet{Chelouche2013};
\citet{CheloucheandZucker2013}; \citet{CheloucheEtal2014};
\citet{EdelsonEtal2015}; \citet{FausnaughEtal2015}). For example, LSST
is expected to deliver BELR line-continuum time delays in
$\sim10^5-10^6$ sources, which is unprecedented when compared to
$\sim50-100$ such measurements conducted via the traditional, yet much
more expensive (per source) spectroscopic method. Sources in the
deep-drilling fields (DDFs) will benefit from the highest quality PRM
time-delay measurements given the factor of $\sim10$ denser sampling.

% --------------------------------------------------------------------

\subsection{Target measurements and discoveries}
\label{sec:\secname:targets}

% Describe the discoveries and measurements you want to make.

% Now, describe their response to the observing strategy. Qualitatively,
% how will the science project be affected by the observing schedule and
% conditions? In broad terms, how would we expect the observing strategy
% to be optimized for this science?

The PRM measurements will probe the size and structure of the
accretion disk and BELR, in a statistical sense, and may provide
improved SMBH mass estimates for sources that have at least
single-epoch spectra. \new{Our goal is to understand the population of
AGN broad line regions, including their geometry. We expect to do this
via  a model that connects the BH mass, BLR geometry and AGN
photometric variability properties via a set of scaling relations. A
simple version of this is could be something like...\newline\newline
So, our target measurements are of $a$ and $b$, the X parameters.
Before we derive a metric that quantifies our ability to measure these
parameters, we can anticipate some of the sensitivity of the
photometric RM method to observing strategy.}

The PRM method is very sensitive to the sampling in each band,
therefore the ability to derive reliable time delays can be affected
significantly by the LSST cadence. The best results will be obtained
by having the most uniform sampling equally for each band.
Additionally, there is a trade-off between the number of DDFs and the
number of time delays that PRM can obtain \citep{CheloucheEtal2014}.
For example, an increase in the number of DDFs, with similarly dense
sampling in each field, can yield a proportionately larger number of
high-quality time delays, down to lower luminosities, but at the
expense of far fewer time delays (for relatively high luminosity
sources) in the main survey.

% --------------------------------------------------------------------

\subsection{Metrics}
\label{sec:\secname:metrics}

Quantifying the response via MAF metrics: definition of the metrics,
and any derived overall figure of merit.


Need to compute the average and the dispersion in the
number of visits, per band, across the sky for the nominal OpSim
(during the entire survey). Since PRM works best for uniform sampling,
need to compare the distributions of the number of visits in each
band, averaged across the sky, and identify ways to minimize any
potential differences between these distributions. By running PRM
simulations, identify the 1) minimum number of visits (in any band)
that can yield any meaningful BELR-continuum lag estimates, and 2) the
largest difference in the number of visits between two different bands
that can yield any meaningful BELR-continuum lag estimates. Repeat
these simulations by doubling the nominal number of DDFs. Assess the
number of meaningful BELR-continuum time delays that can be obtained
with the nominal OpSim, and point out potential perturbations in the
cadence to improve the number and quality of such time delays.

% --------------------------------------------------------------------

\subsection{OpSim Analysis}
\label{sec:\secname:analysis}

OpSim analysis: how good would the default observing strategy be, at
the time of writing for this science project?


% --------------------------------------------------------------------

\subsection{Discussion}
\label{sec:\secname:discussion}

Discussion: what risks have been identified? What suggestions could be
made to improve this science project's figure of merit, and mitigate
the identified risks?


% ====================================================================

\navigationbar
