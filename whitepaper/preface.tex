\setcounter{chapter}{0}
\chapter*{Preface}
\def\chpname{preface}\label{chp:\chpname}
\addcontentsline{toc}{section}{Preface}
\markboth{}{}

\noindent This is a community white paper outlining various science
cases and the impacts that observing strategy will have on them,
quantified using the Metric Analysis Framework. We will describe
various strategies and tradeoffs that impact the observing cadence
(visit sequence), the current cadence baseline, and future directions
for the optimization of the survey strategy. We aim to publish this
white paper on arXiv, and invite community feedback.

The timescale for producing this white paper, started before and
finished after the Observing Strategy workshop at the  August 2015
LSST Project and Community workshop, is many months.

% --------------------------------------------------------------------

\section*{Messages}
% \addcontentsline{toc}{subsection}{Messages}

The main points we aim to convey in this white paper are as follows:

\begin{itemize}

    \item We have a pretty good idea of how we would deploy LSST:
    there is a baseline strategy and a corresponding  simulated visit
    sequence, with which it can be demonstrated that the data required
    for the promised science can be delivered.

    \item However, the baseline strategy is not set in stone, and
    can and will be optimized. Even small changes could
    result in significant improvements to the overall science yield.

    \item These improvements can be predicted via analysis of the
    outputs of the LSST Operations Simulator, \OpSim, using the
    Metric Analysis Framework (\MAF). Once the baseline visit sequence
    has been evaluated with a given science case's metrics, all other
    proposed visit sequences can be compared against it, automatically.

    \item The LSST observing strategy evaluation and optimization
    process will be as open and inclusive as possible. We invite all
    stakeholders to participate.

\end{itemize}

%\raggedright{Project start: July 2015.}
Project start: July 2015.

\clearpage

% --------------------------------------------------------------------

\section*{Guidelines for Authors}
\addcontentsline{toc}{section}{Guidelines for Authors}
\def\secname{guidelines}\label{sec:\secname}

\noindent{\it Phil Marshall}
%(\texttt{@drphilmarshall})

Since this is a community white paper, contributions are welcome from
everyone. Read on for how to make a contribution, and how you should
structure that contribution.

\subsection{How to Get Involved}

The first thing you should do is read and absorb the current version
of the white paper, which you should be able to
\href{http://ls.st/iw2}{view on \GitHub}. (You can also
\href{https://github.com/LSSTScienceCollaborations/ObservingStrategy/raw/master/whitepaper/LSST_Observing_Strategy_White_Paper.pdf}{download
the ``raw'' PDF}, which is hyper-linked for easier navigation.) You
will then be able to provide good feedback, which you should do via
the
\href{https://github.com/LSSTScienceCollaborations/ObservingStrategy/issues}{\GitHub
issues}. Browse the existing issues first: there might be a
conversation you can  join. New issues are most welcome: we'd like to
make this white paper as comprehensive as possible.

To edit the white paper, you'll need to
\href{https://help.github.com/articles/fork-a-repo/}{``fork'' its
repository}. You will then  be able to edit the paper in your own
fork, and when you are ready,  submit a
\href{https://help.github.com/articles/using-pull-requests/}{``pull
request''} explaining what you are doing and the new version that  you
would like to be accepted. It's a good idea to submit this pull
request sooner rather than later, because associated with it will be a
discussion thread that the writing community can use to discuss your
ideas with you. For help getting started with \git and \GitHub, please
see this
\href{https://github.com/drphilmarshall/GettingStarted#top}{handy
guide}.


\subsection{Chapter and Section Design}

For a high-level justification of the following design, please see
\autoref{sec:intro:evaluation}. In short, we're aiming for modular
sections (that are easy to write in parallel, and then re-arrange
within chapters later) focused on one science project each, and
quantified by one (or maybe two) figures of merit (which will likely
depend on other, lower-level metrics). Each section can be thought of
as an observing proposal, demonstrating the performance achievable
given various assumptions about the time awarded.\footnote{These notes
on the white paper design are adapted from the notes at
\href{https://github.com/LSSTScienceCollaborations/ObservingStrategy/blob/master/whitepaper/notes/chapter-template.md}{\texttt{whitepaper/notes/chapter-template.md}}}


The first section of each science chapter needs to be an {\it introduction}
that outlines, very briefly, the commonality of the key science projects
contained in it:  what is to be measured, in broad-brush terms, and
why this is of interest. Then, suppose we were to design an LSST
survey to enable these measurements: qualitatively, what might it look
like, in terms of the choices we are able to make? This chapter
introduction can eventually (when the results are in!) summarize,
again, in very broad brush terms, the results of a number of
investigative sections, one on each science case.

The individual sections following this introduction will then need to
describe the particular discoveries and measurements that are being
targeted in each {\it science case}. It will be helpful to think of a
``science case" as a ``science project" that the section leads {\it
actually plan to do}. Thinking this way means that the sections can
follow the tried and tested format of an observing proposal: a brief
description of the investigation, with references, followed by a
technical feasibility piece.


\subsection{Metric Quantification}

\new{\bf The assessment of each science case will need to be
quantified using the MAF framework, via a set of metrics that need to
be computed for any given observing strategy in order to quantify its
impact on the described science.}

\new{Ideally, these metrics would be combined in a {\it well-motivated
figure of merit}, and used to compare several possible observing
strategies (that have already been simulated with \OpSim). In many (or
perhaps all) cases,  a figure of merit will be a {\it precision} (\ie
a percentage statistical uncertainty) on a astrophysical model
parameter, assuming negligible bias in its inference. Precision is
usually what we need to forecast in order to convince TAC's to give us
telescope time, and so it makes sense to focus on it here too.}

\new{Early on in a metric analysis, it may not be possible to compute
a science case's ideal figure of merit, most likely because to do do
would require a large simulation program to capture the response of
the  parameter measurement to the observing strategy. At this early
stage, it makes sense to look for simple proxies that {\it scale} the
same way as model parameter precision. For example, we might expect
the precision on a set of luminosity function  parameters to scale
with the square root of the number of objects in the sample, and so
$\sqrt{N}$ could be a sensible early figure of merit. Provided we get
the scaling right, {\it we can then compare different observing
strategies by looking  at the percentage change in the figure of
merit,} and arguing that  this will correspond approximately to the
same percentage change in the  ideal case.}

Each science section needs to conclude with a discussion of any risks
that have been identified, and how these could be mitigated. \new{What does
this mean? Each science project will have a {\it threshold acceptable
figure of merit value}, as well as a target (or ``design'') value.  If
an observing strategy gives a figure of merit below the threshold, it
is very important that we know about it.  Optimizing all science cases
in such a complex and diverse set is not really the best way of
thinking about LSST's scheduling task: {\it rather than maximizing
happiness, what we are really trying to do is minimze global
unhappiness (Z.~Ivezic, priv.\ comm.).}}

{\it The comparisons between different simulated
strategies will help make the case for any changes to the baseline
strategy, and in the short term provide motivation for
\href{https://github.com/LSSTScienceCollaborations/ObservingStrategy/blob/master/opsim/README.md}{proposed
new \OpSim simulation runs.}} \new{For some science sections we will
have only a metric design, without an implementation. Subsequent
versions of this white paper will hopefully see these designs realized
and put into action. For now, though, the discussion of risks to these
science cases will necessarily be minimal, containing only predictions
for how the figure of merit will vary among observing strategies.}

\new{Quantitative comparisons between science cases needs will be
carried out in the final chapter, ``Tensions and Trade-offs.'' This
chapter cannot be written until we have quantitative MAF metric
analysis results from the individual science cases to feed into it. At
this early stage, the metrics for each science case, and in
particular, the figures of merit, are not yet well defined. Again,
rather than wait for the ideal figure of merit, {\it we instead
encourage the calculation of simple proxies} that can give at least
some early information about how good the baseline strategy and its
early perturbations will be for each science case.}

The following chapter shows a template introduction and
science case section for you to work from. The latter is checked into
the repository as
\href{https://github.com/LSSTScienceCollaborations/ObservingStrategy/blob/master/whitepaper/section-template.tex}{\texttt{section-template.tex}}.
For an example of a section being developed according to the above guidelines,
please take a look at \autoref{sec:lenstimedelays}.


% ====================================================================

\setcounter{chapter}{-1}
\chapter{Template Science Chapter}
\def\chpname{example}\label{chp:\chpname}

\noindent {\it
Editor Name(s)
}

% --------------------------------------------------------------------

\section{Introduction}
\label{sec:\chpname:intro}

General introduction to the chapter's science projects.

Overview of observing strategy needed by those projects, bringing
out common themes or points of tension.

% --------------------------------------------------------------------

\input{section-template}

% --------------------------------------------------------------------
